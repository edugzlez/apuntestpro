\documentclass{report}
\usepackage[spanish]{babel}
\usepackage[utf8x]{inputenc}
\usepackage[hidelinks]{hyperref}
\author{González Vaquero Eduardo, Lobato de la Cruz Pablo, Muñoz Pérez, Miguel}




\usepackage[margin=1in]{geometry}

\usepackage{fancyhdr}
\pagestyle{fancy}
\rhead{González, Lobato, Muñoz}

\usepackage{amsthm}
\usepackage{amsmath}
\usepackage{amsfonts}
\usepackage{syllogism}
\usepackage{bussproofs}
\usepackage{enumitem}
\usepackage{comment}
\usepackage[T1]{fontenc}
\usepackage{titlesec, blindtext, color}
\usepackage{tikz-cd}



\definecolor{gray75}{gray}{0.75}
\newcommand{\hsp}{\hspace{20pt}}
\titleformat{\chapter}[hang]{\Huge\bfseries}{\thechapter\hsp\textcolor{gray75}{|}\hsp}{0pt}{\Huge\bfseries}

\newtheorem{theorem}{Teorema}[chapter]
\newtheorem{theorem*}[theorem]{*Teorema}
\newtheorem{prop}[theorem]{Proposición}
\newtheorem{lema}[theorem]{Lema}
\newtheorem{cor}[theorem]{Corolario}

\newtheorem{prop*}[theorem]{*Proposición}
\newtheorem{lema*}[theorem]{*Lema}
\newtheorem{cor*}[theorem]{*Corolario}

\theoremstyle{definition}
\newtheorem{definition}[theorem]{Definición}
\newtheorem{definition*}[theorem]{*Definición}
\newtheorem{defs}[theorem]{Definiciones}
\newtheorem{defs*}[theorem]{*Definiciones}
\newtheorem{example}[theorem]{Ejemplo}
\newtheorem{ex}[theorem]{Ejercicio}
\newtheorem{note}[theorem]{Nota}
\newtheorem{sist}[theorem]{Sistema}
\newtheorem*{sist*}{Sistema}
\theoremstyle{remark}
\newtheorem{remark}[theorem]{Nota}

\newtheoremstyle{jeje} % name
    {\topsep}                    % Space above
    {\topsep}                    % Space below
    {}                   % Body font
    {}                           % Indent amount
    {\itshape}                   % Theorem head font
    {.}                          % Punctuation after theorem head
    {.5em}                       % Space after theorem head
    {}  % Theorem head spec (can be left empty, meaning ‘normal’)
\theoremstyle{jeje}
\newtheorem*{sketchproof}{Sketch de demostración}

\DeclareMathOperator{\Tr}{Tr}
\newcommand{\N}{\mathbb{N}}
\newcommand{\Q}{\mathbb{Q}}
\newcommand{\R}{\mathbb{R}}
\newcommand{\K}{\mathbb{K}}
\newcommand{\norm}[1]{\left\|#1\right\|}


\newcommand{\T}{\mathbb{T}}
\renewcommand{\P}{\mathbb{P}}
\newcommand{\C}{\mathbb{C}}
\renewcommand{\H}{\mathbb{H}}
\newcommand{\Z}{\mathbb{Z}}
\newcommand{\F}{\mathbb{F}}
\newcommand{\tr}{\mathrm{tr}}
\newcommand{\idash}{\quad}


\newcommand{\nx}{\texttt{x}}
\newcommand{\ny}{\texttt{y}}
\newcommand{\nz}{\texttt{z}}
\newcommand{\nw}{\texttt{w}}

\newcommand{\n}[1]{\texttt{#1}}
\newcommand{\mc}[1]{\mathcal{#1}}
\newcommand{\fc}[2]{\mc{#1}[\![#2]\!]}
\newcommand{\la}[2]{\langle #1, #2 \rangle}
\newcommand{\lac}[4]{\langle #1, #2 , #3, #4  \rangle}
\newcommand{\dto}{\Rightarrow}
\newcommand{\nn}[1]{\mathrm{#1}}

\newcommand{\State}{\textbf{State}}

\newcommand{\nif}{\texttt{if }}
\newcommand{\nthen}{\texttt{ then }}
\newcommand{\nelse}{\texttt{ else }}
\newcommand{\nskip}{\texttt{skip}}

\newcommand{\nfor}{\texttt{for }}
\newcommand{\nwhile}{\texttt{while }}
\newcommand{\ndo}{\texttt{ do }}
\newcommand{\nend}{\texttt{ end}}
\newcommand{\nvar}{\texttt{var }}
\newcommand{\nrepeat}{\texttt{repeat }}
\newcommand{\nuntil}{\texttt{ until }}
\newcommand{\conf}[3]{\langle #1, #2, #3 \rangle}
%Para poner gorritos:
\usepackage{scalerel,stackengine}
\stackMath
\newcommand\reallywidehat[1]{%
\savestack{\tmpbox}{\stretchto{%
  \scaleto{%
    \scalerel*[\widthof{\ensuremath{#1}}]{\kern.1pt\mathchar"0362\kern.1pt}%
    {\rule{0ex}{\textheight}}%WIDTH-LIMITED CIRCUMFLEX
  }{\textheight}% 
}{2.4ex}}%
\stackon[-6.9pt]{#1}{\tmpbox}%
}


\def\regla#1{\ensuremath\mathbf{(#1)}}

\makeatother


\numberwithin{section}{chapter}
\numberwithin{equation}{chapter}

\makeindex

\begin{document}



\title{Apuntes de Teoría de la Programación}


\author{}

\maketitle




%    Dedication.  If the dedication is longer than a line or two,
%    remove the centering instructions and the line break.
\cleardoublepage
\vspace*{13.5pc}
\begin{center}
\end{center}
\cleardoublepage

%    Change page number to 6 if a dedication is present.
\renewcommand{\contentsname}{Contenidos}
\tableofcontents
\pagenumbering{gobble}

%    Include unnumbered chapters (preface, acknowledgments, etc.) here.



%    Include main chapters here.

\cleardoublepage
\chapter{Introducción}
						
\pagenumbering{arabic}
$\\$

\section{Un ejemplo de la lógica}
La pregunta que motiva todo lo siguiente es: ¿qué es el significado? O, más concretamente, ¿cuál es la relación entre la sintaxis y la semántica? Aunque aquí nos centraremos principalmente en especificar el comportamiento de programas, parece conveniente presentar un ejemplo relacionado con la lógica. Recordemos que, en lógica de primer orden, disponíamos de un método para asignar un valor semántico a cada expresión. En este caso, el valor semántico que nos interesa es la \textit{denotación}, es decir, que por ejemplo $\varphi \lor \psi$ denota lo verdadero en función de $\varphi$ y $\psi$. El método consistía en:
\begin{itemize}
    \item Asumir que las fórmulas atómicas tienen una denotación fija, es decir, podemos determinar previamente si es V o F.
    \item Las denotaciones de $\varphi \lor \psi$, $\neg \varphi$ quedan determinadas por las tablas de verdad correspondientes y el paso anterior.
    \item La denotación de $\forall x. \varphi$ es V si y solo si, para cada $a$ posible, la denotación de $\varphi[a/x]$ es V.
\end{itemize}
Con esto ya sabríamos responder a la primera pregunta que nos hicimos para la lógica de primer orden. Sin embargo, esta aproximación no es la única. Pensemos en el concepto de `prueba' en un sentido computacional. En vez de enfocar el valor semántico hacia la denotación, preguntémonos cuándo un enunciado `tiene una prueba'. Así obtenemos un método alternativo:
\begin{itemize}
    \item Asumir que, para las fórmulas atómicas, conocemos lo que significa una prueba. Por ejemplo, la prueba de que $2 + 2 = 4$ consiste en operar con lápiz y papel.
    \item Una prueba de $\varphi \land \psi$ es un par $(r, s)$ donde $r$ es una prueba de $\varphi$ y donde $s$ es una prueba de $\psi$.
    \item Una prueba de $\varphi \lor \psi$ es un par $(k, r)$, donde o bien $r=0$ y $r$ es prueba de $\varphi$ o bien $k=1$ y $r$ es prueba de $\psi$.
    \item Una prueba de $\varphi \rightarrow \psi$ es una función que lleva pruebas de $\varphi$ en pruebas de $\psi$.
    \item Una prueba de $\neg \varphi$ es una prueba de $\varphi \rightarrow \bot$, donde $\bot$ no admite prueba.
    \item Una prueba de $\forall x. \varphi$ es una función que lleva cada elemento posible $a$ en una prueba de $\varphi[a/x]$.
    \item Una prueba de $\exists x.\varphi$ es un par $(a, r)$ donde $a$ es un elemento posible y $r$ es una prueba de $\varphi[a/x]$.
\end{itemize}
¿Qué ha ocurrido aquí? Hemos dado un valor semántico diferente a la sintaxis lógica usual. Nos encontraremos con situaciones similares a ésta a lo largo del curso pero, en vez de hablar de `proposiciones' y `fórmulas' trataremos `programas'.

\section{Métodos de descripción semántica}

Consideremos un programa como $\nz:= \nx; \nx := \ny; \ny := \nz$. Un análisis sintáctico nos dice que tenemos tres expresiones separadas por $`;$' y que cada una tiene la forma de una variable separada de otra por `$:=$'. El análisis semántico depende en gran medida del sintáctico. Será entonces conveniente asumir programas que están sintácticamente bien escritos, como en este ejemplo. La asignación de un valor semántico se tiene que realizar necesariamente en dos pasos: 
\begin{itemize}
    \item[(i)] Dar un significado expresiones separadas por $`;$'.
    \item[(ii)]  Dar un significado a expresiones formadas por una variable seguida de `$:=$' y una expresión.
\end{itemize}
Nosotros nos centraremos en lo sucesivo en tres enfoques distintos aunque complementarios.

\subsection{Semántica operacional}

Aquí el valor semántico recae sobre el efecto que tiene el programa sobre la máquina en la que se ejecuta, es decir, una descripción operacional os dirá cómo ejecutar el programa que presentamos antes: 
\begin{itemize}
    \item[(i)] Para ejecutar una serie de expresiones separadas por `$;$', las ejecutamos una a una de izquierda a derecha.
    \item[(ii)] Para ejecutar una expresión formada por una variable seguida de `$:=$' y una variable, determinamos el valor de la segunda variable y se lo damos a la primera. 
\end{itemize}
Por tanto, si ejecutamos el programa $\nz:= \nx; \nx := \ny; \ny := \nz$ suponiendo que tenemos las asignaciones iniciales $[\nx \mapsto 5, \ny \mapsto 7, \nz \mapsto 0]$, tenemos:
\begin{align*}
     \langle \nz:= \nx; \nx := \ny; \ny := \nz,  [\nx \mapsto 5, \ny \mapsto 7, \nz \mapsto 0] \rangle & \rightarrow\\
    \langle \nx := \ny; \ny := \nz,  [\nx \mapsto 5, \ny \mapsto 7, \nz \mapsto 5] \rangle & \rightarrow\\
     \langle \ny := \nz,  [\nx \mapsto 7, \ny \mapsto 7, \nz \mapsto 5] \rangle & \rightarrow \\
     [\nx \mapsto 7, \ny \mapsto 5, \nz \mapsto 5]
\end{align*}
Esto es lo que se denomina \textit{semántica operacional estructural}. Pero podemos seguir un procedimiento distinto que muestra menos pasos del proceso anterior:
\begin{prooftree}
    \AxiomC{$\langle \nz := \nx, s_0\rangle \rightarrow s_1 \idash\idash \langle \nx := \ny, s_1\rangle \rightarrow s_2$}
    \UnaryInfC{$\langle \nz := \nx; \nx := \ny, s_0\rangle \rightarrow s_2$}
    \AxiomC{$\langle \ny := \nz, s_2\rangle \rightarrow s_3$}
    \BinaryInfC{$\langle\nz:= \nx; \nx := \ny; \ny := \nz, s_0\rangle \rightarrow s_3$}
\end{prooftree}
Siendo:
$$s_0 := [\nx \mapsto 5, \ny \mapsto 7, \nz \mapsto 0], s_1 := [\nx \mapsto 5, \ny \mapsto 7, \nz \mapsto 5],$$
$$s_2 := [\nx \mapsto 7, \ny \mapsto 7, \nz \mapsto 5], s_3 := [\nx \mapsto 5, \ny \mapsto 5, \nz \mapsto 5].$$
Es decir, aquí hemos resumido toda la información en $\langle e, s\rangle \rightarrow t$, que simboliza el hecho de que, al ejecutar la expresión $e$ en el estado $s$, pasamos al estado $t$.

\subsection{Semántica denotacional}

En este punto de vista, el valor semántico se encuentra en el efecto de cómo se ejecutan los programas. En el caso que tratamos:
\begin{itemize}
    \item[(i)] El efecto de una serie de expresiones separadas por `$;$' consiste en la composición de los efectos de las expresiones.
    \item[(ii)] El efecto de una expresión formada por una variable seguida por `$:=$' y otra variable es una función que lleva un estado en uno nuevo, formado a partir del original haciendo que el valor de la primera variable sea el de la segunda.
\end{itemize}
Es decir, tenemos:
$$\mathcal{S}[\![ \nz:= \nx; \nx := \ny; \ny := \nz ]\!] = \mathcal{S}[\![ \ny := \nz  ]\!]\circ \mathcal{S}[\![ \nx := \ny  ]\!]\circ \mathcal{S}[\![ \nz := \nx  ]\!]$$
Y por tanto, 
\begin{align*}
     & \mathcal{S}[\![ \nz:= \nx; \nx := \ny; \ny := \nz ]\!]([\nx \mapsto 5, \ny \mapsto 7, \nz \mapsto 0]) \\
    = &\mathcal{S}[\![ \ny := \nz  ]\!](\mathcal{S}[\![ \nx := \ny  ]\!](\mathcal{S}[\![ \nz := \nx  ]\!]([\nx \mapsto 5, \ny \mapsto 7, \nz \mapsto 0]))) \\
    = & \mathcal{S}[\![ \ny := \nz  ]\!](\mathcal{S}[\![ \nx := \ny  ]\!]([\nx \mapsto 5, \ny \mapsto 7, \nz \mapsto 5])) \\
    = & \mathcal{S}[\![ \ny := \nz  ]\!]([\nx \mapsto 7, \ny \mapsto 7, \nz \mapsto 5]) \\
    = & [\nx \mapsto 7, \ny \mapsto 5, \nz \mapsto 5].
\end{align*}

Nótese lo que hemos conseguido aquí: hemos traducido el funcionamiento del programa a una serie de objetos matemáticos. Esto será de ayuda en el futuro.

\subsection{Semántica axiomática}

Dado un programa, decimos que es \textit{parcialmente correcto} respecto de una premisa y una consecuencia si, cuando el estado inicial verifica la premisa y el programa termina, el estado final verifica la consecuencia. En nuestro caso, tenemos la propiedad:
$$\{\nx = \texttt{n}\land \ny = \texttt{m}\}\nz:= \nx; \nx := \ny; \ny := \nz\{\ny = \texttt{n}\land \nx = \texttt{m}\}$$
Nótese que esta propiedad no asegura que el programa termine. Desde este punto de vista, lo que queremos es construir un sistema lógico para demostrar la corrección parcial de un programa. En cambio, como veremos, ciertos aspectos de los programas no serán tenidos en cuenta.
La deducción de la corrección parcial del programa de ejemplo es la siguiente:
\begin{prooftree}
    \AxiomC{$\{p_0\}\nz := \nx\{p_1\} \idash\idash \{p_1\}\nx:=\ny\{p_2\}$}
    \UnaryInfC{$\{p_0\}\nz := \nx; \nx := \ny\{p_2\}$}
    \AxiomC{$\{p_2\}\ny := \nz\{p_3\}$}
    \BinaryInfC{$\{p_0\}\nz:= \nx; \nx := \ny; \ny := \nz\{p_3\}$}
\end{prooftree}
Donde
$$p_0 := \texttt{x}=\texttt{n}\land\texttt{y}=\texttt{m}, p_1 := \texttt{z}=\texttt{n}\land\texttt{y}=\texttt{m},$$
$$p_2 := \texttt{z}=\texttt{n}\land\texttt{x}=\texttt{m}, p_3 := \texttt{y}=\texttt{n}\land\texttt{x}=\texttt{m}.$$
Aunque se pueda observar cierta similitud con el enfoque operacional, la diferencia es que aquí trabajamos con aserciones que no tienen en cuenta el funcionamiento del programa. La ventaja de esto consiste en que podemos describir fácilmente determinadas propiedades de tal programa.


\section{El lenguaje While}

Veamos a continuación un ejemplo que iremos desarrollando durante el curso, el lenguaje \textbf{While}. Para especificar las \textit{categorías sintácticas}, especificamos una \textit{metavariable} específica que toma valores en los elementos de cada una:
\begin{itemize}
    \item Numerales: $n\in \mathbf{Num}$.
    \item Variables: $x\in \mathbf{Var}$.
    \item Expresiones aritméticas: $a\in \mathbf{Aexp}$.
    \item Expresiones booleanas: $b\in \mathbf{Bexp}$.
    \item Sentencias: $S\in \mathbf{Stm}$.
\end{itemize}
En caso de que hiciera falta emplear más de una metavariable, emplearemos, por ejemplo, $n', n'', \dots$ y $n_1, n_2, \dots$. Supondremos que las categorías $\mathbf{Num}$ y $ \mathbf{Var}$ se construyen de manera natural. Las categorías sintácticas $\mathbf{Aexp}$, $\mathbf{Bexp}$ y $\mathbf{Stm}$ se construyen de la siguiente forma:
\[
    \begin{array}{l}
         a ::= n\ |\ x\ |\ a_1+ a_2\ |\ a_1\times a_2\ |\ a_1-a_2 \\
         b ::= \n{true}\ |\ \n{false}\ |\ a_1 = a_2\ |\ a_1 \leq a_2\ |\ \neg b\ |\ b_1 \land b_2 \\
         S ::= x:=a\ |\ \n{skip}\ |\ S_1;S_2\ |\ \n{if}\ b\ \n{then}\ S_1\ \n{else}\ S_2\ |\ \n{while}\ b\ \n{do}\ S
         
    \end{array}
\]


Si volvemos al ejemplo de antes, la expresión $\n{z} := \n{x}; \n{x} := \n{y}; \n{y} := \n{z}$, podemos definir una \textit{sintaxis concreta}, en el sentido de que, de los diferentes árboles de derivación para tal expresión, debemos escoger uno. En cambio, lo que acabamos de definir arriba es un ejemplo de \textit{sintaxis abstracta}, y los árboles de derivación posibles son todos distintos elementos de la categoría $\mathbf{Stm}$. De hecho, en caso de que queramos expresar la prioridad de las operaciones, lo haremos mediante el uso de paréntesis.

\section{Semántica para expresiones}

Veamos, a modo de ejemplo, cómo dar significado a los numerales. Expresaremos la gramática de $\mathbf{Num}$ por
\[
    n ::= \n{0}\ |\ \n{1} \ |\ n\ \n{0} \ |\ n\ \n{1}
\]

Intepretaremos los numerales como si fuesen la expresión binaria de un número natural. Es decir, se tendrá una \textit{aplicación semántica} $\mc{N}: \mathbf{Num} \to \Z$ dada recursivamente por:
\[
    \begin{array}{l}
         \fc{\mc{N}}{\n{0}} = 0 \\
         \fc{\mc{N}}{\n{1}}= 1 \\
         \fc{\mc{N}}{n \n{0}} = 2\otimes \fc{\mc{N}}{n}\\
         \fc{\mc{N}}{n \n{1}} = 2\otimes \fc{\mc{N}}{n} \oplus 1
    \end{array}
\]
Donde $\oplus, \otimes$ expresan las operaciones de suma y producto en $\Z$ y $0, 1, 2$ los enteros correspondientes: la distinción entre objetos sintácticos y semánticos tiene que ser cuidadosa. Las igualdades anteriores se llaman \textit{ecuaciones semánticas}, nos indican cómo asociar un objeto matemático a un símbolo. Además, tenemos aquí un ejemplo de lo que se llama el \textit{principio de composición}, es decir, construimos el significado de una expresión (\textit{elemento compuesto}) en función del de sus componentes (\textit{elementos base}). Ésto facilita aplicar el método de demostración general que seguiremos, la \textit{inducción estructural}, que consiste en primero demostrar una propiedad para cada elemento base y después demostrarla para los compuestos empleando la hipótesis de inducción. Veamos un ejemplo a continuación.

Primero recordemos que una función $f: A \rightarrow B$ es \textit{parcial} si hay elementos $a\in A$ en los que no está definida. En caso contrario, se denomina \textit{función total}.

\begin{prop}
La función $\mc{N}: \mathbf{Num} \to \Z$ es total.
\end{prop}
\begin{proof}[Demostración]
Por inducción. De los siguientes casos, $(a), (b)$ son los \texttit{casos base} y $(c), (d)$ los \textit{inductivos}:
\begin{itemize}
    \item[(a)] Si $n = \n{0}$, $\fc{N}{n} = 0$.
    \item[(b)] Si $n = \n{1}$, $\fc{N}{n} = 1$.
    \item[(c)] Si $n = m\n{0}$, $\fc{N}{n} = \fc{N}{m\n{0}} = 2 \otimes \fc{N}{m}$, y por hipótesis de inducción tenemos el resultado.
    \item[(d)] Análogo a (c).
\end{itemize}
\end{proof}

\begin{example}
Definamos una gramática y una semántica asociada para interpretar los numerales (binarios), de modo que el primer caracter de cada cadena represente el signo (positivo o negativo) del número representado por el resto de la misma. Una gramática posible es:
\[
    n ::= \n{0}\ |\ \n{1} \ |\ \n{0}n \ |\ \n{1}n
\]
Definiremos su semántica atendiendo a que las cadenas $\n{0}$ y $\n{1}$ representan el número 0, pues se compondrían solo del signo, sin acompañar ningún número. \\ \\ 
La función semántica es $\mc{S}: \mathbf{Num} \rightarrow \Z$, definida por:
\[
    \begin{array}{l}
        \fc{S}{0} = 0\\
        \fc{S}{1} = 0 \\
         \fc{\mc{S}}{\n{0}n} = \fc{N}{n} \\
         \fc{\mc{S}}{\n{1}n} =-\fc{N}{n}\\
    \end{array}
\]
Notemos el siguiente detalle: en principio, la función $\mc{N}$, como tal, no puede tomar valores del modo anterior. Sin embargo, se puede probar que, como la gramática que dimos en la definición de $\mc{N}$ y la que hemos escrito arriba generan las mismas cadenas, la función $\mc{N}$ se puede identificar fácilmente con la función que buscamos. \\ \\
Pese a ello es posible definir dicha función de la siguiente forma
\[
    \begin{array}{l}
         \fc{N}{\n{0}} = 0 \\
         \fc{N}{\n{1}} = 1 \\
         \fc{N}{\n{0}n} = \fc{AUX}{n, 0} \\
         \fc{N}{\n{1}n} =  \fc{AUX}{n, 1} \\
    \end{array}
\]
y la función $\mc{AUX}: \mathbf{Num} \times \mathbb{N} \to \mathbb{N} $
\[
    \begin{array}{l}
         \fc{AUX}{\n{0}, x} = 2\otimes x \\
         \fc{AUX}{\n{1}, x} = 2\otimes x \oplus 1 \\
         \fc{AUX}{\n{0}n, x} = \fc{AUX}{n, 2\otimes x} \\
         \fc{AUX}{\n{1}n, x} =  \fc{AUX}{n, 2\otimes x \oplus 1} \\
    \end{array}
\]
donde el segundo parámetro representa un acumulador. Por construcción de la gramática la lectura de la cadena se hace de izquierda a derecha, dificultando un poco la construcción de la semántica.
\end{example}




Desde la perspectiva de la semántica denotacional, el significado de una expresión está determinado por los valores que toman en ella las variables. Esto motiva el concepto de \textit{estado}, definido en nuestro caso como un elemento del conjunto $\mathbf{State} := \mathbf{Var} \rightarrow \Z$, es decir, como una función que lleva una variable en su valor (un entero positivo). Por tanto, el significado de la expresión viene dado por una función auxiliar $\mc{A}: \mathbf{Aexp} \rightarrow (\mathbf{State}\rightarrow \Z)$, donde $\mathcal{A}$ toma una exprexión aritmética y un estado $s$\footnote{Nótese que $\mathcal{A}$ nos lleva $a$ a una función $\fc{A}{a}$ y aplicamos tal función a $s$ escribiendo $ \fc{A}{a}s$. Por otro lado, con $s\, x$ nos referimos a $s$ aplicado a $x$.}:
\begin{align*}
    \fc{A}{n}s & = \fc{N}{n} \\
    \fc{A}{x}s & = s\ x \\ 
    \fc{A}{a_1 + a_2}s & = \fc{A}{a_1}s \oplus\fc{A}{a_2}s \\
    \fc{A}{a_1 \times a_2}s & = \fc{A}{a_1}s \otimes \fc{A}{a_2}s \\
     \fc{A}{a_1 - a_2}s & = \fc{A}{a_1}s \ominus \fc{A}{a_2}s 
\end{align*}


\begin{example}
Podemos añadir a la definición la ecuación semántica
$$\fc{A}{-a}s = 0 \ominus \fc{A}{a}s$$
Incluso podemos prescindir del $0$ por cómo está definida $\ominus$. En cambio,
$$\fc{A}{-a}s  = \fc{A}{\n{0}-a}s$$
no está definida de forma composicional.
\end{example}

Se puede repetir el procedimiento anterior para definir una función semántica para los booleanos, $\mc{B} : \mathbf{Bexp} \rightarrow (\mathbf{State}\rightarrow Bool)$, siendo $Bool := \{\mathbf{tt}, \mathbf{ff}\}$, definida por:
\begin{align*}
    \fc{B}{\n{true}}s & = \mathbf{tt}\\
    \fc{B}{\n{false}}s & = \mathbf{ff} \\
    \fc{B}{a_1 = a_2}s & = \begin{cases} \mathbf{tt}\text{, si }\fc{A}{a_1}s\text{ es igual a } \fc{A}{a_2}s \\ \mathbf{ff}\text{, en otro caso}\end{cases}\\
    \fc{B}{a_1 \leq a_2}s & = \begin{cases} \mathbf{tt}\text{, si }\fc{A}{a_1}s\text{ es menor que } \fc{A}{a_2}s \\ \mathbf{ff}\text{, en otro caso}\end{cases}\\
    \fc{B}{\neg b}s & = \begin{cases} \mathbf{tt}\text{, si }\fc{B}{b}s\text{ es } \mathbf{ff} \\ \mathbf{ff} \text{, en otro caso}\end{cases}\\
    \fc{B}{b_1 \land b_2}s & = \begin{cases} \mathbf{tt}\text{, si }\fc{B}{b_1}s\text{ es } \mathbf{tt}\text{ y } \fc{B}{b_2}s\text{ es } \mathbf{tt} \\ \mathbf{ff}\text{, en otro caso}\end{cases}
\end{align*}

De nuevo, por inducción estructural, es fácil demostrar el siguiente resultado, que es análogo al que vimos para los numerales:
\begin{prop}
La función $\mc{B} : \mathbf{Bexp} \rightarrow (\mathbf{State}\rightarrow Bool)$ es total.
\end{prop}

El siguiente ejemplo ilustra cómo podemos extender una categoría sintáctica (de forma cuidadosa):
\begin{example}
Consideremos la extensión $\mathbf{Bexp}'$ de $\mathbf{Bexp}$:
$$b \,\, ::= \, \, \n{true} \, | \, \n{false} \, | \, a_1 = a_2 \, | \, a_1 \neq a_2 \, | \, a_1 \leq a_2 \, | \, a_1 \geq a_2 \, | \, a_1 < a_2 \, | \, a_1 > a_2 \, | \, \neg b \, | \, b_1 \land b_2 \, | \, b_1 \lor b_2 \, | \, b_1 \Rightarrow b_2 \, | \, b_1\Leftrightarrow b_2$$
Dos expresiones booleanas $b_1, b_2$ se dicen \textit{equivalentes} si, para cada estado $s$, $\fc{B}{b_1}s = \fc{B}{b_2}s$. Veamos que, dada una expresión $b' \in \mathbf{Bexp}'$, existe $b \in \mathbf{Bexp}$ equivalente a $b'$. La demostración consiste en dos pasos: (i) Dar un valor semántico a cada expresión de la extensión, (ii) Comprobar que podemos expresar el valor semántico de $b'$ mediante $b$, empleando las igualdades sintácticas naturales.
\begin{itemize}
    \item Si $b'$ es una expresión de $\mathbf{Bexp}$, $b := b'$.
    \item Si $b'$ es de la forma $a_1 \neq a_2$, tomamos $b$ como $\neg(a_1 = a_2)$.
    \item Si $b'$ es de la forma $a_1 \geq a_2$, tomamos $b$ como $a_2 \leq a_1$.
    \item Si $b'$ es de la forma $a_1 < a_2$, tomamos $b$ como $(a_1 \leq a_2)\land \neg(a_1 = a_2)$.
    \item Si $b'$ es de la forma $a_1 > a_2$, tomamos $b$ como $(a_2 \leq a_1)\land \neg(a_1 = a_2)$.
    \item Si $b'$ es de la forma $b_1 \lor b_2$, tomamos $b$ como $\neg(\neg b_1 \land \neg b_2)$.
    \item Si $b'$ es de la forma $b_1 \Rightarrow b_2$, tomamos $b$ como $\neg(b_1 \land \neg b_2)$.
    \item Si $b'$ es de la forma $b_1 \Leftrightarrow b_2$, tomamos $b$ como $\neg(b_1 \land \neg b_2) \land \neg(b_2 \land \neg b_1)$.
\end{itemize}
Notemos que podríamos haber razonado inductivamente, pero para mayor claridad hemos indicado cuál es la traducción concreta de cada expresión.
\end{example}

\section{Propiedades semánticas}

En esta sección introducimos dos conceptos fundamentales que son comunes a la lógica.

\subsection{Variables libres}

Dada una expresión aritmética $a$, su conjunto de \textit{variables libres}, $\mathrm{FV}(a) \subseteq \mathbf{Var}$, se define composicionalmente como:
\begin{align*}
    \mathrm{FV}(n) & = \emptyset\\
    \mathrm{FV}(x) & = \{x\}\\
    \mathrm{FV}(a_1 + a_2) & = \mathrm{FV}(a_1) \cup \mathrm{FV}(a_2) \\
    \mathrm{FV}(a_1 \times a_2) & = \mathrm{FV}(a_1) \cup \mathrm{FV}(a_2) \\
    \mathrm{FV}(a_1 - a_2) & = \mathrm{FV}(a_1) \cup \mathrm{FV}(a_2) 
\end{align*}
El siguiente resultado nos dice que $\mathrm{FV}(a)$ determina el valor semántico de $a$:

\begin{lema}
Sea $a\in \mathbf{Aexp}$. Sean $s, s' \in \mathbf{State}$ tales que, para cada $x \in \mathrm{FV}(a)$, $s\ x = s'\x$. Entonces $\fc{A}{a}s = \fc{A}{a}s'$.
\end{lema}
\begin{proof}[Demostración]
Veamos los casos base:
\begin{itemize}
    \item Si $a:= n$, sabemos que $\fc{A}{a}s := \fc{N}{n} =: \fc{A}{a}s'$.
    \item Si $a := x$, entonces, como $x \in \mathrm{FV}(a)$, por hipótesis tenemos que $\fc{A}{a}s := s\ x  = s'\ x := \fc{A}{n}s'$.
\end{itemize}
Los casos inductivos son:
\begin{itemize}
    \item Si $a$ es de la forma $a_1 + a_2$, $\fc{A}{a}s := \fc{A}{a_1}s + \fc{A}{a_2}s$ y $\fc{A}{a}s' := \fc{A}{a_1}s' + \fc{A}{a_2}s'$. Como $\mathrm{FV}(a_i) \subseteq \mathrm{FV}(a_1) \cup\mathrm{FV}(a_2) = \mathrm{FV}(a_1 + a_2)$, por la hipótesis de inducción aplicada a $a_i$, tenemos que $\fc{A}{a_i}s = \fc{A}{a_i}s'$, para $i = 1, 2$. Entonces,
$$\fc{A}{a_1+a_2}s = \fc{A}{a_1}s + \fc{A}{a_2}s = \fc{A}{a_1}s' + \fc{A}{a_2}s' = \fc{A}{a_1+a_2}s',$$
como queríamos. 
    \item Para $a_1 * a_2$ y $a_1 - a_2$ basta repetir lo anterior (ya que el conjunto de variables libres es el mismo).
\end{itemize}
\end{proof}

De la misma forma, para expresiones booleanas, tenemos:
\begin{align*}
    \mathrm{FV}(\n{true}) & = \emptyset\\
    \mathrm{FV}(\n{true}) & = \emptyset\\
    \mathrm{FV}(a_1 = a_2) & = \mathrm{FV}(a_1) \cup \mathrm{FV}(a_2) \\
    \mathrm{FV}(a_1 \leq a_2) & = \mathrm{FV}(a_1) \cup \mathrm{FV}(a_2) \\
    \mathrm{FV}(\neg b) & = \mathrm{FV}(b)\\
    \mathrm{FV}(b_1 \land b_2) & = \mathrm{FV}(b_1) \cup \mathrm{FV}(b_2) 
\end{align*}

La demostración del anterior lema se puede repetir de nuevo:

\begin{lema}
Sea $b\in \mathbf{Bexp}$. Sean $s, s' \in \mathbf{State}$ tales que, para cada $x \in \mathrm{FV}(b)$, $s\ x = s'\x$. Entonces $\fc{B}{b}s = \fc{B}{b}s'$.
\end{lema}
\begin{proof}[Demostración]
Casos base:
\begin{itemize}
    \item Si $b := \n{true}$, $\fc{B}{b}s := V =: \fc{B}{b}s'$ y análogamente con $\n{false}$.
    \item Si $b$ es de la forma $a_1 = a_2$, con $a_1, a_2 \in \mathbf{Aexp}$, sabemos que
$$\fc{B}{a_1 = a_2}s  = \begin{cases} V\text{, si }\fc{A}{a_1}s\text{ es igual a } \fc{A}{a_2}s \\ F\text{, en otro caso}\end{cases} \text{ y que }\, \fc{B}{a_1 = a_2}s' = \begin{cases} V\text{, si }\fc{A}{a_1}s'\text{ es igual a } \fc{A}{a_2}s' \\ F\text{, en otro caso}\end{cases}$$
Como suponemos que para cada $x \in \mathrm{FV}(b)$, $s\ x = s'\ x$ y $\mathrm{FV}(a_1), \mathrm{FV}(a_2) \subseteq \mathrm{FV}(b)$, se sigue que se verifican las hipótesis del lema anterior, y que por tanto $\fc{A}{a_i}s = \fc{A}{a_i}s'$, para $i = 1, 2$. Ahora bien, $\fc{B}{b}s$ es $V$ si y solo si $\fc{A}{a_1}s$ es igual a $\fc{A}{a_2}s$ y, por lo que acabamos de decir, esto es cierto si y solo si $\fc{A}{a_1}s'$ es igual a $\fc{A}{a_2}s'$, que es precisamente equivalente a que $\fc{B}{b}s'$ sea $V$.
    \item Si $b$ es de la forma $a_1 \leq a_2$, con con $a_1, a_2 \in \mathbf{Aexp}$, el procedimiento es análogo al anterior.
\end{itemize}

Para los casos inductivos tenemos:
\begin{itemize}
    \item Si $b$ es de la forma $\neg b'$, para cierta $b' \in \mathrm{Bexp}$, sabemos que entonces $\mathrm{FV}(b) = \mathrm{FV}(b')$. Como suponemos que, para cada $x \in \mathrm{FV}(b)$, $s\ x = s' \ x$, podemos aplicar la hipótesis de inducción, y entonces obtenemos que $\fc{B}{b}s$ es $V$ si y solo si $\fc{B}{b'}s = \fc{B}{b'}s'$ es $V$, que es equivalente a que $\fc{B}{b}s'$ sea $V$. Por tanto, $\fc{B}{b}s = \fc{B}{b}s'$.
    \item Si $b$ es de la forma $b_1 \land b_2$, para ciertas $b_1, b_2 \in \mathrm{Bexp}$, sabemos que entonces $\mathrm{FV}(b) =  \mathrm{FV}(b_1)\cup\mathrm{FV}(b_2)$ y, siguiendo los razonamientos que ya hemos hecho antes, podemos aplicar la hipótesis de inducción sobre $b_1$ y $b_2$. Entonces $\fc{B}{b}s$ es $V$ si y solo si $\fc{B}{b_1}s$ es $V$ y $\fc{B}{b_2}s$ es $v$, que equivale a que $\fc{B}{b_1}s'$ sea $V$ y $\fc{B}{b_2}s'$ sea $V$, que es cierto si y solo si $\fc{B}{b}s'$ es $V$ y, por tanto, $\fc{B}{b}s = \fc{B}{b}s'$.
\end{itemize}
\end{proof}

\subsection{Sustitución}

Si tenemos dos expresiones aritméticas $a, a_0$ y $x\in \mathrm{FV}(a)$, entonces denotamos por $a[x \mapsto a_0]$ a la expresión obtenida al \textit{sustituir} cada ocurrencia de $x$ en $a$ por $a_0$. Se define composicionalmente como:
\begin{align*}
    n[x \mapsto a_0] & = n \\
    y[x \mapsto a_0] & = \begin{cases}a_0 \text{, si } x=y\\ y\text{, si } x\neq y\end{cases}\\
    (a_1+a_2)[x \mapsto a_0] & = a_1[x \mapsto a_0] + a_2[x \mapsto a_0]\\
    (a_1\times a_2)[x \mapsto a_0] & = a_1[x \mapsto a_0] \times a_2[x \mapsto a_0]\\
    (a_1-a_2)[x \mapsto a_0] & = a_1[x \mapsto a_0] - a_2[x \mapsto a_0]
\end{align*}
También podemos definir la sustitución en relación con los estados:
$$(s[y\mapsto v])x := \begin{cases}v\text{, si } x=y\\ s\ x \text{, si } x\neq y\end{cases}$$

La relación entre ambos conceptos se muestra en el siguiente resultado:
\begin{lema}
Dadas $a, a_0 \in \mathbf{Aexp}$, para todo $s \in \mathbf{State}$ se cumple que $$\fc{A}{a[y\mapsto a_0]}s = \fc{A}{a}(s[y\mapsto \fc{A}{a_0}s]).$$
\end{lema}
\begin{proof}[Demostración]
De nuevo, una demostración rutinaria por inducción estructural. Los casos base son los siguientes:
\begin{itemize}
    \item Si $a := n$, $\fc{A}{a[y\mapsto a_0]}s =  \fc{A}{n}s = \fc{N}{n} = \fc{A}{a}([s\mapsto \fc{A}{a_0}s])$.
    \item Si $a:= x$, entonces
    \[
        \fc{A}{a[y\mapsto a_0]}s = \left\{\begin{array}{ll}
            \fc{A}{a_0}s & x=y \\
            \fc{A}{x}s & x\neq y
        \end{array}\right.
    \]
    $$
        \fc{A}{a}(s[y\mapsto \fc{A}{a_0}s]) = (s[y\mapsto \fc{A}{a_0}s])\ x = \left\{ \begin{array}{ll}
             \fc{A}{a_0}s & x=y \\
             s\ x & x\neq y
        \end{array}\right.
    $$
    \item Si $a := a_1 + a_2$ con $a_1,a_2$ cumpliendo la proposición. Se tiene 
    \[
        \fc{A}{a_i[y\mapsto a_0]}s = \fc{A}{a_i}(s[y\mapsto \fc{A}{a_0}s]) =
        \fc{A}{a_i}s' 
    \]
    para $i\in\{1, 2\}$. Se denota $s' := (s[y\mapsto \fc{A}{a_0}s])$.
    Entonces
    \begin{eqnarray*}
        \fc{A}{(a_1+a_2)[y\mapsto a_0]}s &=&
        \fc{A}{a_1[y\mapsto a_0]+a_2[y\mapsto a_0]}s  \\
        &=& \fc{A}{a_1[y\mapsto a_0]}s\oplus\fc{A}{a_2[y\mapsto a_0]}s \\
        &\overset{hip.ind.}{=}& \fc{A}{a_1}s' \oplus\fc{A}{a_2}s' \\
        &=& \fc{A}{a_1+a_2}
    \end{eqnarray*}
    \item El caso $a := a_1 \times a_2$ es análogo.
\end{itemize}
\end{proof}


Todo lo anterior justifica una noción que será importante a lo largo del curso. Dada una categoría sintáctica $\textbf{Cat}$, dos expresiones $b_1, b_2 \in \textbf{Cat}$ y una función semántica $\mc{C}: \textbf{Cat}\rightarrow C$, se dice que $b_1, b_2$ son \textit{semánticamente equivalentes} si para todo $s\in \textbf{State}$ se tiene
\[
    \fc{C}{b_1}s = \fc{C}{b_2}s.
\]

\cleardoublepage
\chapter{Semántica operacional}
						
$\\$

En el anterior capítulo hemos visto cómo dar un valor semántico al lenguaje While mediante el punto de vista de la semántica denotacional. Centrémonos ahora en la semántica operacional. La distinción fundamental que hacíamos de este enfoque es la siguiente:
\begin{itemize}
    \item Semántica operacional \textit{natural}, que describe cómo se han obtenido los resultados generales de las ejecuciones.
    \item Semántica operacional \textit{estructural}, que describe cómo se ha obtenido cada paso en la ejecución.
\end{itemize}
Para ambos tipos de semántica operacional, el valor semántico de cada expresión será especificado por un \textit{sistema de transiciones}, compuesto de dos configuraciones distintas:
\begin{itemize}
    \item[] $\langle S, s\rangle$, que denota que la expresión $S$ se ejecutará desde el estado $s$.
    \item[] $s$, que denota un estado terminal. Las \textit{configuraciones terminales} tendrán esta forma.
\end{itemize}
Finalmente, es necesaria una \textit{relación de transición} que describa cómo tiene lugar la ejecución. La diferencia entre las dos semánticas se encuentra principalmente en ésta. De hecho, veremos que ambos tipos de semántica son, en cierto sentido, equivalentes.

\section{Semántica operacional natural}


\subsection{Sistema de transiciones}

La relación de transición $\langle S, s\rangle \rightarrow s'$ se puede leer como que, la ejecución de $S$ desde el estado $s$ terminará y el nuevo estado será $s'$. Está determinada por las siguientes reglas\footnote{Nótese que las variables $S_1, S_2, s, s', s''$, etc. son libres.}:  


\begin{sist*}[$\nn{While}_\nn{ns}$]\mbox{}
\begin{itemize}
    \item[] $[\text{ass}_{\text{ns}}]$

        \begin{center}
              \centerAlignProof
              \quad
              \centerAlignProof
                \AxiomC{}
                \UnaryInfC{ $\langle x:= a, s\rangle \rightarrow s[x\mapsto \fc{A}{a}s]$}
              \DisplayProof
        \end{center}

    \item[] $[\text{skip}_{\text{ns}}]$ 
        \begin{center}
              \centerAlignProof
              \quad
              \centerAlignProof
                \AxiomC{}
                \UnaryInfC{$\la{\n{skip}}{s}\rightarrow s$}
                  \DisplayProof
        \end{center}

    \item[][$\text{comp}_{\text{ns}}$]
        \begin{center}
              \centerAlignProof
                
              \quad
              \centerAlignProof
                \AxiomC{$\la{S_1}{s} \to s'$}
                \AxiomC{$\la{S_2}{s'} \to s''$}
                \BinaryInfC{$\la{S_1; S_2}{s} \to s''$}
              \DisplayProof
        \end{center}
\newpage
\item[][$\text{if}^{\text{tt}}_{\text{ns}}$]


\begin{center}
      \centerAlignProof
        
      \quad
      \centerAlignProof
        \AxiomC{$\la{S_1}{s} \to s'$}
        \UnaryInfC{$\la{\n{if}\ b\ \n{then}\ S_1\ \n{else}\ S_2}{s} \to s'$}
      \DisplayProof
      \quad
      \centerAlignProof
        $\text{ si }\fc{B}{b}s = \mathbf{tt}$
\end{center}
\item[] [$\text{if}^{\text{ff}}_{\text{ns}}$]

\begin{center}
      \centerAlignProof
       
      \quad
      \centerAlignProof
        \AxiomC{$\la{S_2}{s} \to s'$}
        \UnaryInfC{$\la{\n{if}\ b\ \n{then}\ S_1\ \n{else}\ S_2}{s} \to s'$}
      \DisplayProof
      \quad
      \centerAlignProof
        $\text{ si }\fc{B}{b}s = \mathbf{ff}$
\end{center}
\item[][$\text{while}^{\text{tt}}_{\text{ns}}$]

\begin{center}
      \centerAlignProof
        
      \quad
      \centerAlignProof
        \AxiomC{$\la{S}{s} \to s'$}
        \AxiomC{$\la{\n{while}\ b\ \n{do}\ S}{s'} \to s''$}
        \BinaryInfC{$\la{\n{while}\ b\ \n{do}\ S}{s} \to s''$}
      \DisplayProof
      \quad
      \centerAlignProof
        $\text{ si }\fc{B}{b}s = \mathbf{tt}$
\end{center}
\item[] [$\text{while}^{\text{ff}}_{\text{ns}}$]

\begin{center}
      \centerAlignProof
       
      \quad
      \centerAlignProof
        \AxiomC{}
        \UnaryInfC{$\la{\n{while}\ b\ \n{do}\ S}{s} \to s$}
      \DisplayProof
      \quad
      \centerAlignProof
        $\text{ si }\fc{B}{b}s = \mathbf{ff}$
\end{center}
\end{itemize}
\end{sist*}

Aclaremos un poco la terminología:

\begin{definition}
Una \textit{regla} en general tiene la forma general
\begin{center}
      \centerAlignProof
       
      \quad
      \centerAlignProof
        \AxiomC{$\la{S_1}{s_1}\rightarrow s'_1 \dots \la{S_n}{s_n}\rightarrow s'_n $}
    \UnaryInfC{$\la{S}{s}\rightarrow s'$}
      \DisplayProof
      \quad
      \centerAlignProof
    $\text{ si } \varphi$
\end{center}
donde los términos que aparecen encima y bajo la línea son, respectivamente, las \textit{premisas} y la \textit{conclusión}, y donde $\varphi$ es la \textit{condición}. Cuando empleamos las reglas anteriores para obtener una transición $\la{S}{s}\rightarrow s'$, obtenemos un \textit{árbol de derivación}. Una regla sin premisas se llama \textit{axioma}.
\end{definition}

Consideremos el problema de construir un árbol de derivación para una expresión $S$ y un estado $s$. El método general consiste en partir de la `raíz' y encontrar las `hojas', es decir, el paso inicial consiste en buscar una regla de modo que su conclusión tenga la ejecución $\la{S}{s}$ en su parte izquierda. Los pasos inductivos son:
\begin{itemize}
    \item Si la regla encontrada es un axioma, entonces podemos determinar el estado terminal y terminamos.
    \item Si la regla encontrada no es un axioma, entonces el siguiente paso consiste en buscar un árbol de derivación para sus premisas. 
\end{itemize}
Nótese que, en cada paso, las condiciones para aplicar cada regla tienen que ser verificadas. En el futuro demostraremos algo que parece falso a primera vista: que en el lenguaje While hay a lo sumo un árbol de derivación posible para cada ejecución $\la{S}{s}$.


\begin{definition}
Decimos que una ejecución de la expresión $S$ desde el estado $s$, $\la{S}{s}$, \textit{termina} si existe un estado $s'$ tal que $\la{S}{s}\rightarrow s'$. Si tal estado no existe entonces decimos que la ejecución \textit{cicla}. Para una expresión $S$, decimos que \textit{siempre termina} si $\la{S}{s}$ termina para cada elección de $s$ y que \textit{siempre cicla} si $\la{S}{s}$ cicla para cada elección de $s$.
\end{definition}

\begin{example}
Podemos tratar de determinar si las siguientes expresiones terminan o ciclan siempre:
\begin{enumerate}
    \item $\n{while} \, \neg(\n{x}=1) \, \n{do} \, (\n{y} := \n{y} \times \n{x}; \n{x} := \n{x} - 1)$.
    \item $\n{while} \, 1 \leq \n{x} \, \n{do} \, (\n{y} := \n{y} \times \n{x}; \n{x} := \n{x} - 1)$.
    \item $\n{while true do skip}$.
\end{enumerate}
La primera para si se inicializa $x$ con un valor mayor o igual que $1$ y cicla en caso contrario: nótese que si $\n{x}<1$ decrecerá continuamente, luego es imposible que la condición del bucle llegue a no cumplirse.\\ \\
La segunda para siempre. Es parecida a la anterior salvo por el hecho de que la condición del bucle hace que, para $\n{x}<1$, se pare. \\ \\
La tercera dejaría intacta la configuración inicial. Sin embargo, la ejecución cicla pese a no realizar ninguna acción. Esto es porque la condición del bucle es siempre cierta.
\end{example}

\subsection{Propiedades}

El sistema de transición nos da un entorno en el que estudiar las propiedades de las expresiones. Veamos a continuación una definición precisa de un concepto que introdujimos al final de la introducción:
\begin{definition}
Dos expresiones $S_1, S_2$ se dicen \textit{semánticamente equivalentes} si para cada par $s, s' \in \mathbf{State}$, 
$$\la{S_1}{s}\rightarrow s' \text{ si y solo si } \la{S_2}{s}\rightarrow s'.$$
\end{definition}

\begin{lema}\label{lemans}
\normalfont $\n{while } b \n{ do } S$ \textit{es semánticamente equivalente a} $\n{if } b \n{ then } (S; \n{while } b \n{ do } S)\n{ else skip}$.
\end{lema}
\begin{proof}
Dividimos la prueba en dos implicaciones:


\noindent\textit{Parte 1.} Supongamos que se cumple $\la{\n{while } b \n{ do } S}{s}\rightarrow s''$. Entonces existe un árbol de derivación para él, $T$. $T$ puede tener dos formas en función de la regla que hayamos aplicado: o bien hemos aplicado la regla o el axioma [$\text{while}^{\text{ff}}_{\text{ns}}$]. Veamos cada caso:
\\

\noindent$(a)$ Si hemos aplicado la regla [$\text{while}^{\text{tt}}_{\text{ns}}$], $T$ es de la forma:
\begin{center}
      \centerAlignProof
      \quad
      \centerAlignProof
      \AxiomC{$\dots$}
      \UnaryInfC{$\la{S}{s}\rightarrow s'$}
      \AxiomC{$\dots$}
      \UnaryInfC{$\la{\n{while}\ b\ \n{do}\ S}{s'} \to s''$}
    \BinaryInfC{$\la{\n{while } b \n{ do } S}{s}\rightarrow s''$}[$\text{while}^{\text{tt}}_{\text{ns}}$]
      \DisplayProof
      \quad
      \centerAlignProof
      \end{center}

con $\fc{B}{b}s = \mathbf{tt}$. Ahora bien, notemos que:
\begin{center}
      \centerAlignProof
      \quad
      \centerAlignProof
      \AxiomC{$\dots$}
      \UnaryInfC{$\la{S}{s}\rightarrow s'$}
      \AxiomC{$\dots$}
      \UnaryInfC{$\la{\n{while}\ b\ \n{do}\ S}{s'} \to s''$}
    \BinaryInfC{$\la{S;\n{while } b \n{ do } S}{s}\rightarrow s''$}$[\text{comp}_{\text{ns}}]$
      \DisplayProof
      \quad
      \centerAlignProof
      \end{center}
Usando que $\fc{B}{b}s = \mathbf{tt}$, podemos aplicar:
\begin{center}
      \centerAlignProof
      \quad
      \centerAlignProof
    \AxiomC{$\la{S;\n{while } b \n{ do } S}{s}\rightarrow s''$}
    \UnaryInfC{$\la{\n{if}\ b\ \n{then}\ (S;\n{while } b \n{ do } S)\ \n{else}\ \n{skip}}{s} \to s''$}[$\text{if}^{\text{ff}}_{\text{ns}}$]
      \DisplayProof
      \quad
      \centerAlignProof
      \end{center}
Y por tanto, $\la{\n{if}\ b\ \n{then}\ (S;\n{while } b \n{ do } S)\ \n{else}\ \n{skip}}{s} \to s''$.
\\

\noindent $(b)$ Si hemos aplicado la regla  [$\text{while}^{\text{ff}}_{\text{ns}}$], $T$ es de la forma:
\begin{center}
      \centerAlignProof
       
      \quad
      \centerAlignProof
        \AxiomC{}
        \UnaryInfC{$\la{\n{while}\ b\ \n{do}\ S}{s} \to s$}
      \DisplayProof
      \quad
      \centerAlignProof
\end{center}
es decir, necesariamente $s=s''$ y $\fc{B}{b}s = \mathbf{ff}$. Usando el axioma $[\text{skip}_{\text{ns}}]$, directamente obtenemos que 
\begin{center}
      \centerAlignProof
       
      \quad
      \centerAlignProof
        \AxiomC{}
        \UnaryInfC{$\la{\n{skip}}{s}\rightarrow s''$}$[\text{skip}_{\text{ns}}]$
      \DisplayProof
      \quad
      \centerAlignProof
\end{center}
Pero entonces, 
\begin{center}
      \centerAlignProof
      \quad
      \centerAlignProof
        \AxiomC{$\la{\n{while}\ b\ \n{do}\ S}{s} \to s$}
        \AxiomC{$\la{\n{skip}}{s}\rightarrow s''$}
        \BinaryInfC{$\la{\n{if}\ b\ \n{then}\ (S;\n{while } b \n{ do } S)\ \n{else}\ \n{skip}}{s} \to s''$}[$\text{if}^{\text{ff}}_{\text{ns}}$]
      \DisplayProof
      \quad
      \centerAlignProof
\end{center}
y por tanto obtenemos el resultado.
\\

\noindent\textit{Parte 2.} Supongamos ahora que se cumple $\la{\n{if}\ b\ \n{then}\ (S;\n{while } b \n{ do } S)\ \n{else}\ \n{skip}}{s} \to s''$. Entonces, tenemos un árbol de derivación $T$ y, de nuevo, podemos distinguir qué forma tendrá según las reglas que hayamos aplicado:
\\

\noindent $(a)$ Si hemos aplicado la regla [$\text{if}^{\text{tt}}_{\text{ns}}$], $T$ es de la forma:
\begin{center}
      \centerAlignProof
        
      \quad
      \centerAlignProof
      \AxiomC{$\dots$}
        \UnaryInfC{$\la{S; \n{while } b \n{ do } s}{s} \to s''$}
        \UnaryInfC{$\la{\n{if}\ b\ \n{then}\ (S;\n{while } b \n{ do } S)\ \n{else}\ \n{skip}}{s} \to s''$}[$\text{if}^{\text{tt}}_{\text{ns}}$]
      \DisplayProof
      \quad
      \centerAlignProof
\end{center}
y con $\fc{B}{b}s = \mathbf{tt}$. Ahora bien, solo hemos podido obtener la premisa anterior mediante $[\text{comp}_{\text{ns}}]$, por tener una expresión de la forma $S_1; S_2$ en la ejecución. Entonces deducimos que $T$ es:
\begin{center}
      \centerAlignProof
      \quad
      \centerAlignProof
        \AxiomC{$\dots$}
        \UnaryInfC{$\la{S}{s}\rightarrow s'$}
        \AxiomC{$\dots$}
        \UnaryInfC{$\la{\n{while } b \n{ do } S}{s'}\rightarrow s''$}
        \BinaryInfC{$\la{S; \n{while } b \n{ do } s}{s} \to s''$}[$\text{comp}_{\text{ns}}$]
      \DisplayProof
      \quad
      \centerAlignProof
\end{center}
Pero entonces notemos que, usando la hipótesis $\fc{B}{b}s = \mathbf{tt}$:
\begin{center}
      \centerAlignProof
      \quad
      \centerAlignProof
        \AxiomC{$\dots$}
        \UnaryInfC{$\la{S}{s}\rightarrow s'$}
        \AxiomC{$\dots$}
        \UnaryInfC{$\la{\n{while } b \n{ do } S}{s'}\rightarrow s''$}
        \BinaryInfC{$\la{\n{while}\ b\ \n{do}\ S}{s} \to s''$}[$\text{while}^{\text{tt}}_{\text{ns}}$]
      \DisplayProof
      \quad
      \centerAlignProof
\end{center}
y obtenemos el resultado.
\\

\noindent $(b)$ Si hemos usado la regla [$\text{if}^{\text{ff}}_{\text{ns}}$], deducimos que $\fc{B}{b}s = \mathbf{ff}$ y que por tanto tenemos un árbol de derivación para $\la{\n{skip}}{s}\rightarrow s''$ y, por tanto, que $s=s''$. Pero usando  [$\text{while}^{\text{ff}}_{\text{ns}}$], tenemos el resultado (el razonamiento ha sido análogo al apartado $(b)$ de la Parte 1).

\end{proof}



\begin{example}
Veamos que $S_1; (S_2; S_3)$ y $(S_1; S_2); S_3$ son semánticamente equivalentes. Si suponemos que $\la{S_1; (S_2; S_3)}{s}\rightarrow s'$, entonces es porque en su árbol de derivación hemos empleado [$\text{comp}_{\text{ns}}$] a las premisas $\la{S_1}{s}\rightarrow s''$ y $\la{S_2;S_3}{s''}\rightarrow s'$. A su vez, la segunda premisa proviene del mismo modo de las premisas $\la{S_2}{s''}\rightarrow t$ y $\la{S_3}{t}\rightarrow s'$. Es decir, tenemos las siguientes hojas:
\begin{itemize}
    \item[(a)] $\la{S_1}{s}\rightarrow s''$.
    \item[(b)] $\la{S_2}{s''}\rightarrow t$.
    \item[(c)] $\la{S_3}{t}\rightarrow s'$.
\end{itemize}
Ahora, combinando (a) y (b) con [$\text{comp}_{\text{ns}}$], obtenemos $\la{S_1;S_2}{s}\rightarrow t$ y, combinando esto con (c) de la misma forma, obtenemos que $\la{(S_1; S_2);S_3}{s}\rightarrow s'$, como queríamos ver. La otra implicación es análoga.

Notemos, por otro lado, que en general $S_1;S_2$ y $S_2; S_1$ no son semánticamente equivalentes: si tratásemos de hacer lo mismo que antes, obtendríamos las hojas $\la{S_1}{s}\rightarrow s''$ y $\la{S_2}{s''}\rightarrow s'$ por un lado y $\la{S_2}{s}\rightarrow s''$ y $\la{S_1}{s''}\rightarrow s'$ por otro, y en general no hay forma de combinar cada par de premisas para obtener la conclusión deseada.
\end{example}
\begin{example}\textbf{}
Podemos expandir el sistema $\text{While}_\text{ns}$ el siguiente modo: añadimos dos reglas que permitan dar una semántica de la expresión $\n{for }x:= a_1 \n{ to } a_2\n{ do }S$, es decir, 
\begin{itemize}
    \item[] 
\begin{center}
      \centerAlignProof
      [$\text{for}_{\text{ns}}^{\text{tt}}$]
      \quad
      \centerAlignProof
        \AxiomC{$\la{x:= a_1;S}{s}\rightarrow s'$}
        \AxiomC{$\la{\n{for } \n{x}:= \n{x}+\n{1 to } a_2 \n{ do }S}{s'}\to s''$}
        \BinaryInfC{$\la{\n{for } \n{x}:= a_1\n{ to } a_2 \n{ do }S}{s}\to s''$}
        \DisplayProof
      \quad
      \centerAlignProof
        $\text{ si }\fc{B}{a_1\leq a_2}s = \mathbf{tt}$
\end{center}
 \item[] 
\begin{center}
      \centerAlignProof
      [$\text{for}_{\text{ns}}^{\text{ff}}$]
      \quad
      \centerAlignProof
        \AxiomC{}
        \UnaryInfC{$\la{\n{for } \n{x}:= a_1\n{ to } a_2 \n{ do }S}{s}\to s[\n{x}\mapsto \fc{A}{a_1}]s$}
        \DisplayProof
      \quad
      \centerAlignProof
        $\text{ si }\fc{B}{a_1\leq a_2}s = \mathbf{ff}$
\end{center}
\end{itemize}
Pero debemos tener un especial cuidado con este tipo de reglas, por ejemplo, podemos descuidar que en $a_1$ aparezca la variable $\n{y}$, a saber, que $a_1$ contenga $\n{y} + 3$, y que por otro lado en $S$ tengamos $\n{y}=5$. Del mismo modo, podríamos tener que la variable $\n{x}$ ya aparece del mismo modo como $\n{x} = 4$, por ejemplo. Si $\n{x}$ apareciera en $a_2$ entonces también tendríamos este problema.
\end{example}

Aunque no lo demostraremos, se puede observar que el sistema $\text{While}_\text{ns}$ es \textit{Turing-completo}, es decir, en él podemos simular cualquier computación posible en una máquina de Turing. Por tanto, se podía pensar que podemos introducir reglas para ciertas expresiones en función de su correlato en $\text{While}_\text{ns}$ (que existe, por lo anterior). Sin embargo, si quisiéramos introducir $\n{for }x:= a_1 \n{ to } a_2\n{ do }S$ como un bucle $\n{while ... do ...}$, acabaríamos teniendo apariciones de $\n{while ... do ...}$ en las reglas asociadas a $\n{for }x:= a_1 \n{ to } a_2\n{ do }S$, lo que difiere de la semántica operacional que hemos visto hasta ahora.
\\

\begin{example}
Podríamos extender el lenguaje While con dos reglas para la expresión $\n{repeat } S \n{ until } b$:
\begin{itemize}
    \item[] 
\begin{center}
      \centerAlignProof
      [$\text{repeat}_{\text{ns}}^{\text{tt}}$]
      \quad
      \centerAlignProof
        \AxiomC{$\la{S}{s}\rightarrow s'$}
        \UnaryInfC{$\la{\n{repeat } S \n{ until } b}{s}\rightarrow s'$}
        \DisplayProof
      \quad
      \centerAlignProof
        $\text{ si }\fc{B}{b}s' = \mathbf{tt}$
\end{center}
    \item[] 
\begin{center}
      \centerAlignProof
      [$\text{repeat}_{\text{ns}}^{\text{ff}}$]
      \quad
      \centerAlignProof
        \AxiomC{$\la{S}{s}\rightarrow s'$}
        \AxiomC{$\la{\n{repeat } S \n{ until } b}{s'}\rightarrow s''$}
        \BinaryInfC{$\la{\n{repeat } S \n{ until } b}{s}\rightarrow s''$}
        \DisplayProof
      \quad
      \centerAlignProof
        $\text{ si }\fc{B}{b}s' = \mathbf{ff}$
\end{center}
\end{itemize}
\end{example}
\begin{prop}
Son semánticamente equivalentes:
\begin{itemize}
    \item $\n{repeat } S \n{ until } b$.
    \item $S; \n{if } b \n{ then skip else } (\n{repeat } S \n{ until } b)$.
\end{itemize}
\end{prop}
\begin{proof}

\noindent\textit{Parte 1.} Supongamos que $\la{\n{repeat } S \n{ until } b}{s}\rightarrow s'$. Solo tenemos las siguientes posibilidades:
\\

\noindent\textit{(a)} Si hemos aplicado la regla [$\text{repeat}_{\text{ns}}^{\text{tt}}$], tenemos:
\begin{center}
      \centerAlignProof
      [$\text{repeat}_{\text{ns}}^{\text{tt}}$]
      \quad
      \centerAlignProof
        \AxiomC{$\la{S}{s}\rightarrow s'$}
        \UnaryInfC{$\la{\n{repeat } S \n{ until } b}{s}\rightarrow s'$}
        \DisplayProof
      \quad
      \centerAlignProof
        $\text{ si }\fc{B}{b}s' = \mathbf{tt}$
\end{center}
Ahora bien, por otro lado, podemos aplicar el axioma [$\text{skip}_{\text{ns}}$] para obtener directamente que $\la{\n{skip}}{s'}\rightarrow s'$. Ahora, como $\text{ si }\fc{B}{b}s' = \mathbf{tt}$, podemos aplicar la regla [$\text{if}^{\text{tt}}_{\text{ns}}$]:
\begin{center}
      \centerAlignProof
      \quad
      \centerAlignProof
        \AxiomC{$\la{\n{skip}}{s'}\rightarrow s'$}
        \UnaryInfC{$\la{\n{if}\ b\ \n{then skip else}\ (\n{repeat } S \n{ until } b)}{s'} \to s'$}[$\text{if}^{\text{tt}}_{\text{ns}}$]
      \DisplayProof
      \quad
      \centerAlignProof
\end{center}
Y, entonces, 
        \begin{center}
              \centerAlignProof
              \quad
              \centerAlignProof
                \AxiomC{$\la{S}{s} \to s'$}
                \AxiomC{$\la{\n{if}\ b\ \n{then skip else}\ (\n{repeat } S \n{ until } b)}{s'} \to s'$}
                \BinaryInfC{$\la{S; \n{if } b \n{ then skip else } (\n{repeat } S \n{ until } b)}{s} \to s'$}[$\text{comp}_{\text{ns}}$]
              \DisplayProof
        \end{center}
Luego obtenemos el resultado.
\\

\noindent\textit{(b)} Si hemos aplicado la regla [$\text{repeat}_{\text{ns}}^{\text{ff}}$], tenemos:
\begin{center}
      \centerAlignProof
      [$\text{repeat}_{\text{ns}}^{\text{ff}}$]
      \quad
      \centerAlignProof
        \AxiomC{$\la{S}{s}\rightarrow s''$}
        \AxiomC{$\la{\n{repeat } S \n{ until } b}{s''}\rightarrow s'$}
        \BinaryInfC{$\la{\n{repeat } S \n{ until } b}{s}\rightarrow s'$}
        \DisplayProof
      \quad
      \centerAlignProof
        $\text{ si }\fc{B}{b}s' = \mathbf{ff}$
\end{center}
Ahora, usando que $\text{ si }\fc{B}{b}s' = \mathbf{ff}$, 
\begin{center}
      \centerAlignProof
      \quad
      \centerAlignProof
        \AxiomC{$\la{\n{repeat } S \n{ until } b}{s''}\rightarrow s'$}
        \UnaryInfC{$\la{\n{if}\ b\ \n{then skip else}\ (\n{repeat } S \n{ until } b)}{s''} \to s'$}[$\text{if}^{\text{ff}}_{\text{ns}}$]
      \DisplayProof
      \quad
      \centerAlignProof
\end{center}

Pero entonces, 
\begin{center}
              \centerAlignProof
              \quad
              \centerAlignProof
                \AxiomC{$\la{S}{s} \to s''$}
                \AxiomC{$\la{\n{if}\ b\ \n{then skip else}\ (\n{repeat } S \n{ until } b)}{s''} \to s'$}
                \BinaryInfC{$\la{S; \n{if } b \n{ then skip else } (\n{repeat } S \n{ until } b)}{s} \to s'$}[$\text{comp}_{\text{ns}}$]
              \DisplayProof
\end{center}
\noindent\textit{Parte 2.} Supongamos que $\la{S;\n{ if } b \n{ then } \n{skip } \n{else } \n{repeat } S \n{ until } b}{s}\to s'$. La única posibilidad es haber aplicado la regla $[\text{comp}_{\text{ns}}]$
\begin{center}
    \centerAlignProof
    \quad
    \centerAlignProof
    \AxiomC{$\la{S}{s} \to s_0$}
    \AxiomC{$\la{\n{if}\ b\ \n{then skip else}\ (\n{repeat } S \n{ until } b)}{s_0} \to s'$}
    \BinaryInfC{$\la{S;\n{if } b \n{ then } \n{skip } \n{else } \n{repeat } S \n{ until } b}{s}\to s'$}
    \DisplayProof
\end{center}
para algún $s_0 \in \textbf{State}$. Para la transición $\la{\n{if}\ b\ \n{then skip else}\ (\n{repeat } S \n{ until } b)}{s_0} \to s'$ tenemos dos posibilidades
\\ \\
\noindent\textit{(a)} Si $\fc{B}{b}s_0 = \textbf{tt}$ entonces únicamente existe la posibilidad de que se haya derivado de $[\text{if}^\nn{tt}_\nn{ns}]$:
\begin{center}
    \centerAlignProof
    \quad
    \centerAlignProof
    \AxiomC{$\la{skip}{s_0} \to s' $}
    \UnaryInfC{$\la{\n{if}\ b\ \n{then skip else}\ (\n{repeat } S \n{ until } b)}{s_0} \to s'$}
    \DisplayProof
\end{center}
y la única forma de que sea cierto $\la{skip}{s_0} \to s'$ es que $s_0 = s'$. Como se verifica $\la{S}{s} \to s_0$ entonces se verifica $\la{S}{s} \to s'$ y se puede aplicar la regla $[\text{repeat}^\nn{tt}_\nn{ns}]$:
\begin{center}
    \centerAlignProof
    \quad
    \centerAlignProof
    \AxiomC{$\la{S}{s} \to s' $}
    \UnaryInfC{$\la{\n{repeat } S \n{ until } b}{s} \to s'$}
    \DisplayProof
\end{center}
obteniendo el resultado pues ya se sabe que $\la{S}{s} \to s_0$.
\\ \\
\noindent\textit{(b)}  Si $\fc{B}{b}s_0 = \textbf{ff}$ entonces solo cabe la posibilidad de que haya partido de $[\text{if}^\nn{ff}_\nn{ns}]$:

\begin{center}
    \centerAlignProof
    \quad
    \centerAlignProof
    \AxiomC{$\la{\n{repeat } S \n{ until } b}{s_0} \to s'$}
    \UnaryInfC{$\la{\n{if}\ b\ \n{then skip else}\ (\n{repeat } S \n{ until } b)}{s_0} \to s'$}
    \DisplayProof
\end{center}
teniendo así $\la{\n{repeat } S \n{ until } b}{s_0} \to s'$ y entonces se puede deducir
\begin{center}
    \centerAlignProof
    \quad
    \centerAlignProof
    \AxiomC{$\la{S}{s} \to s_0$}
    \AxiomC{$\la{\n{repeat } S \n{ until } b}{s_0} \to s'$}
    \BinaryInfC{$\la{\n{repeat } S \n{ until } b}{s} \to s'$}
    \DisplayProof
\end{center}
mediante $[\text{repeat}^\nn{ff}_\nn{ns}]$ pues $\fc{B}{b}s_0 = \textbf{ff}$. \\
\end{proof}


Para poder demostrar que una propiedad como la anterior se verifica en árboles sencillos y compuestos, emplearemos la demostración por \textit{inducción sobre reglas}, que se compone de dos pasos:
\begin{enumerate}
    \item Primero comprobamos que la propiedad se verifica para los axiomas del sistema.
    \item Para cada regla, suponiendo que las premisas verifican la propiedad , comprobamos que también se cumple para la conclusión (siempre y cuando se verifiquen las condiciones de la regla).
\end{enumerate}
El siguiente resultado nos dice que, en general, hay \textit{una} manera de deducir una configuración mediante las reglas del sistema de transición $\nn{While}_\nn{ns}$:

\begin{theorem}\label{determinismo}
El sistema de transiciones $\nn{While}_{\nn{ns}}$ es determinista, es decir, para cada $S \in \mathbf{Stm}$, $s, s', s'' \in \mathbf{State}$,  
$$\la{S}{s}\rightarrow s' \text{ y } \la{S}{s}\rightarrow s'' \text{ implica que } s'= s''.$$
\end{theorem}
\begin{proof}
Para simplificar la demostración, vamos a definir una propiedad sintáctica de las reglas del sistema $\nn{While}_\nn{ns}$. Decimos que dos reglas son \textit{independientes entre sí} cuando no es posible obtener una mediante la aplicación de la otra. Notemos que este es el caso de nuestro sistema: las reglas $[\nn{while}_\nn{ns}^\nn{tt}]$ y $[\nn{while}_\nn{ns}^\nn{ff}]$ son independientes entre sí porque ambas tienen premisas distintas (suponemos que $\mathbf{tt}$ y $\mathbf{ff}$ son distintos). Entonces, como cada regla es independiente de la otra (y evidentemente, cada regla es determinista), deducimos que, en caso de que tengamos $\la{S}{s}\rightarrow s'$ y $\la{S}{s}\rightarrow s''$, necesariamente tendremos que haber aplicado la misma única regla posible en los dos casos para llegar a las respectivas configuraciones. Es fácil convencerse entonces de que, por inducción sobre las reglas, la propiedad deseada se cumple\footnote{Obviamente, esa demostración no es intercambiable con la demostración formal por inducción estructural. Podría demostrarse el caso general del que hemos hablado, a saber, formalizando lo que significa precisamente la independencia de dos reglas.}. 
\end{proof}
\begin{example} Podemos añadir una semántica $\n{forVar}\ x\ \n{do}\ S$ que ejecute la sentencia $S$ siempre que $x$ sea distinto de $0$ y lo incremente en $1$ en cada iteración. Veamos que sería semánticamente equivalente a $\n{while}\ \neg (x=0)\ \n{do}\ (S; x:=x+1)$.
\\ \\
Primero, definimos la semantica de $\n{forVar}\ x\ \n{do}\ S$:
        \begin{center}
              \centerAlignProof
              \quad
              \centerAlignProof
                \AxiomC{}
                \LeftLabel{$[\text{for}^0]$ }
                \RightLabel{si $\fc{A}{x}s = 0$}
                \UnaryInfC{$\la{\n{forVar}\ x\ \n{do}\ S}{s} \to s$}
              \DisplayProof
        \end{center}
        
        \begin{center}
              \centerAlignProof
              \quad
              \centerAlignProof
                \LeftLabel{$[\text{for}^{\ne 0}]$ }
                \RightLabel{si $\fc{A}{x}s \ne 0$}
                \AxiomC{ $\la{S; x := x+1}{s} \to s'$ }
                \AxiomC{$\la{\n{forVar}\ x\ \n{do}\ S}{s'} \to s_1$}
                \BinaryInfC{$\la{\n{forVar}\ x\ \n{do}\ S}{s} \to s_1$}
              \DisplayProof
        \end{center}

\noindent Veamos que $\la{\n{forVar}\ x\ \n{do}\ S}{s} \to s_1$ implica $\la{\n{while}\ \neg (x=0)\ \n{do}\ (S; x:=x+1)}{s}\to s_1 $.\\
Para empezar, sabemos que $\fc{A}{x}s = 0$ si y solo si $\fc{B}{\neg (x=0) }s = \mathbf{ff}$, dividimos la demostración en dos pasos:

\begin{enumerate}
    \item Si $x=0$ tenemos por $[\text{for}^0]$ que:
    \begin{center}
              \centerAlignProof
              \quad
              \centerAlignProof
                \AxiomC{$\la{\n{forVar}\ x\ \n{do}\ S}{s} \to s$}
              \DisplayProof
        \end{center}
        
    Como $\fc{B}{\neg (x=0) }s = \mathbf{ff}$ por la regla [$\text{while}^{\text{ff}}_{\text{ns}}$] sabemos que:

    \begin{center}
          \centerAlignProof
       
          \quad
          \centerAlignProof
            \AxiomC{$\la{\n{while}\ \neg (x=0)\ \n{do}\ (S; x:=x+1)}{s} \to s$}
          \DisplayProof
          \quad
          \centerAlignProof
    \end{center}    
    
    \item Si $ x\ne0$, entonces suponemos ciertas las siguientes premisas:
    \begin{enumerate}[label=\alph*)]
        \item  \AxiomC{$\la{S; x := x+1}{s} \to s_2$} \DisplayProof
        \item   \AxiomC{$\la{\n{forVar}\ x\ \n{do}\ S}{s_2} \to s'$} \DisplayProof
    \end{enumerate}
    
    
    pues la transición $\la{\n{forVar}\ x\ \n{do}\ S}{s} \to s'$ solo puede haber provenido de:
    
    \begin{center}
              \centerAlignProof
              \quad
              \centerAlignProof
                
                \AxiomC{ $\la{S; x := x+1}{s} \to s_2$ }
                \LeftLabel{$[\text{for}^{\ne 0}]$}
                \AxiomC{$\la{\n{forVar}\ x\ \n{do}\ S}{s_2} \to s'$}
                \BinaryInfC{$\la{\n{forVar}\ x\ \n{do}\ S}{s} \to s'$}
              \DisplayProof
        \end{center}
    Podemos aplicar la hipótesis de inducción sobre $\la{\n{forVar}\ x\ \n{do}\ S}{s_2} \to s'$ y por lo tanto tenemos que $\la{\n{forVar}\ x\ \n{do}\ S}{s_2} \to s'$ implica que $\la{\n{while}\ \neg (x=0)\ \n{do}\ (S; x:=x+1)}{s_2} \to s' $, luego podemos costruir el siguiente árbol de derivación:
    
    \begin{center}
              \centerAlignProof
              \quad
              \centerAlignProof
                
                \AxiomC{ $\la{S; x := x+1}{s} \to s_2$ }
                \LeftLabel{ [$\text{while}^{\text{tt}}_{\text{ns}}$]}
                \AxiomC{$\la{\n{while}\ \neg (x=0)\ \n{do}\ (S; x:=x+1)}{s_2} \to s'$}
                \BinaryInfC{$\la{\n{while}\ \neg (x=0)\ \n{do}\ (S; x:=x+1)}{s} \to s'$}
              \DisplayProof
        \end{center}
\end{enumerate}
Supongamos ahora que $\la{\n{while}\ \neg (x=0)\ \n{do}\ (S; x:=x+1)}{s}\to s'$. Entonces, distinguimos los siguientes casos:
\begin{enumerate}
    \item Si hemos aplicado [$\text{while}^{\text{ff}}_{\text{ns}}$], entonces 
    \begin{center}
      \centerAlignProof
       
      \quad
      \centerAlignProof
        \AxiomC{}
        \UnaryInfC{$\la{\n{while}\ \neg (x=0)\ \n{do}\ (S; x:=x+1)}{s} \to s'$}[$\text{while}^{\text{ff}}_{\text{ns}}$]
      \DisplayProof
      \quad
      \centerAlignProof
\end{center}
y además deducimos que $s=s'$ y que $x \neq 0$. Pero entonces tenemos que, directamente:
\begin{center}
              \centerAlignProof
              \quad
              \centerAlignProof
                \AxiomC{}
                \LeftLabel{$[\text{for}^0]$ }
                \RightLabel{}
                \UnaryInfC{$\la{\n{forVar}\ x\ \n{do}\ S}{s} \to s$}
              \DisplayProof
        \end{center}
es decir, obtenemos la implicación deseada.
    \item Si hemos aplicado [$\text{while}^{\text{tt}}_{\text{ns}}$], 
\begin{center}
      \centerAlignProof
        
      \quad
      \centerAlignProof
        \AxiomC{$\la{(S; x:=x+1)}{s} \to s'$}
        \AxiomC{$\la{\n{while}\  \neg (x=0)\ \n{do}\ (S; x:=x+1)}{s'} \to s''$}
        \BinaryInfC{$\la{\n{while}\ \neg (x=0)\ \n{do}\ (S; x:=x+1)}{s} \to s''$}[$\text{while}^{\text{tt}}_{\text{ns}}$]
      \DisplayProof
      \quad
      \centerAlignProof
\end{center}
y además deducimos que $x \neq 0$. Si aplicamos hipótesis de inducción sobre $\la{\n{while}\  \neg (x=0)\ \n{do}\ (S; x:=x+1)}{s'} \to s''$, obtenemos que $\la{\n{forVar}\ x\ \n{do}\ S}{s'} \to s''$. Pero entonces, juntando las premisas anteriores,  
  \begin{center}
              \centerAlignProof
              \quad
              \centerAlignProof
                \LeftLabel{$[\text{for}^{\ne 0}]$ }
                \RightLabel{}
                \AxiomC{ $\la{S; x := x+1}{s} \to s'$ }
                \AxiomC{$\la{\n{forVar}\ x\ \n{do}\ S}{s'} \to s''$}
                \BinaryInfC{$\la{\n{forVar}\ x\ \n{do}\ S}{s} \to s''$}
              \DisplayProof
        \end{center}
luego obtenemos el resultado.
\end{enumerate}
\end{example}
\subsection{Expresiones}

Finalmente, podemos definir el valor semántico de cada $S \in \mathbf{Stm}$ mediante una aplicación
$\mc{S}_\nn{ns}: \mathbf{Stm} \rightarrow (\mathbf{State}\hookrightarrow\mathbf{State})$, donde
\begin{align*}
    \mc{S}_\nn{ns}[[S]]:  \mathbf{State} & \hookrightarrow  \mathbf{State} \\
                     s               & \mapsto          \begin{cases} s' \text{, si } \la{S}{s}\to s' \\ 
                        \nn{indefinido} \text{, en otro caso}
                        \end{cases}
\end{align*}
El determinismo de $\nn{While}_\nn{ns}$ implica que está bien definida. Además, es parcial porque, como vimos, la expresión $\n{while true do skip}$ siempre entra en bucle, es decir, $ \mc{S}_\nn{ns}[[\n{while true do skip}]]s = \mathrm{indefinido}$, para cada $s \in \mathbf{State}$.


\begin{example}
Podemos definir, por ejemplo, una semántica de paso largo para $\mathbf{Aexp}$ mediante la relación de transición $\la{a}{s}\rightarrow_A z$, donde $\la{a}{s}$ significa que $a \in \mathbf{Aexp}$ se evalúa en $s\in \mathbf{State}$ y $z\in \Z$ es un estado final:
\begin{sist*}[$\mathbf{Aexp}_\nn{ns}$]\mbox{}
\begin{itemize}
    \item[]
\begin{prooftree}
    \AxiomC{}
    \LeftLabel{}
    \RightLabel{$\text{ si } n \in \mathbf{Num}$}
    \UnaryInfC{$\la{n}{s}\rightarrow_A \fc{N}{n}$}
\end{prooftree}
    \item[]
    
\begin{prooftree}
    \AxiomC{}
    \RightLabel{$\text{ si } x \in \mathbf{Var}$}
    \UnaryInfC{$\la{x}{s}\rightarrow_A s x$}
\end{prooftree}

\item[]
 \begin{prooftree}
    \AxiomC{$\la{a_1}{s}\rightarrow_A z_1$}
    \AxiomC{$\la{a_2}{s}\rightarrow_A z_2$}
    \LeftLabel{}
    \RightLabel{}
    \BinaryInfC{$\la{a_1 \n{ op } a_2}{s}\rightarrow_A z_1 * z_2$}
\end{prooftree}
$\text{donde } \n{op} \text{ se refiere a } \oplus, \ominus, \otimes \text{ y } * \text{ a } +, -, \times$.
\end{itemize}
\end{sist*}
\end{example}

\begin{prop}
Sean $a \in \mathbf{Aexp}$, $s \in \mathbf{State}$, $z \in \Z$. Entonces $\la{a}{s}\rightarrow_A z$ si y solo si $\fc{A}{a}s =z$.
\end{prop}
\begin{proof}
La demostración es por inducción estructural. Los casos base son:
\begin{itemize}
    \item Si $a = n$, entonces $\la{a}{s}\rightarrow_A \fc{N}{n} = \fc{A}{n}s$.
    \item Si $a = x$, entonces $\la{a}{s}\rightarrow_A sx = \fc{A}{x}s$.
\end{itemize}
Resumimos los casos inductivos en:
\begin{itemize}
    \item Si $a = a_1 \n{ op } a_2$, entonces, empleando la última regla de $\mathbf{Aexp}_\nn{ns}$, tenemos que $\la{a_i}{s}\rightarrow_A z_i$ si y solo si (por hipótesis de inducción) $\fc{A}{a_i}s = z_i$, con $i = 1, 2$. Pero entonces sabemos que $\fc{A}{a_1 \n{ op }a_2}S = \fc{A}{a_1}s * \fc{A}{a_2}s = z_1 * z_2 = z$.
\end{itemize}
\end{proof}

\begin{example}
Siguiendo el ejemplo anterior, podemos definir un sistema de transiciones para expresiones booleanas como sigue. Definimos $\la{b}{s}\rightarrow_B X$, donde $\la{b}{s}$ indica que $b$ se evalúa en el estado $s$ y donde $X\in Bool$. 
\begin{sist*}[$\mathbf{Bexp}_\nn{ns}$]\mbox{}
\begin{itemize}
    \item[]
    \begin{prooftree}
    \AxiomC{}
    \LeftLabel{}
    \UnaryInfC{$\la{\n{true}}{s}\rightarrow_B \mathbf{tt}$}
\end{prooftree}
\item[]
    \begin{prooftree}
    \AxiomC{}
    \LeftLabel{}
    \UnaryInfC{$\la{\n{false}}{s}\rightarrow_B \mathbf{ff}$}
\end{prooftree}
    \item[]
 \begin{prooftree}
    \AxiomC{$\la{a_1}{s}\rightarrow_A z_1$}
    \AxiomC{$\la{a_2}{s}\rightarrow_A z_2$}
    \LeftLabel{}
    \RightLabel{}
    \BinaryInfC{$\la{a_1 = a_2}{s}\rightarrow_B X$}
    \end{prooftree}
    donde $X$ es el booleano correspondiente (véase la definición de $\fc{B}{\cdot}$).

\item[]
\begin{prooftree}
    \AxiomC{$\la{a_1}{s}\rightarrow_A z_1$}
    \AxiomC{$\la{a_2}{s}\rightarrow_A z_2$}
    \LeftLabel{}
    \RightLabel{}
    \BinaryInfC{$\la{a_1 \leq s a_2}{s}\rightarrow_B X$}
        \end{prooftree}
    donde $X$ es el booleano correspondiente (de nuevo, empleando la definición de $\fc{B}{\cdot}$).
\item[]\begin{prooftree}
    \AxiomC{$\la{a_1}{s}\rightarrow_A z_1$}
    \AxiomC{$\la{a_2}{s}\rightarrow_A z_2$}
    \LeftLabel{}
    \RightLabel{}
    \BinaryInfC{$\la{a_1 \leq s a_2}{s}\rightarrow_B X$}
    \end{prooftree}
        donde $X$ es el booleano correspondiente.
\item[]
\begin{prooftree}
    \AxiomC{$\la{b}{s}\rightarrow_B X$}
    \LeftLabel{}
    \RightLabel{}
    \UnaryInfC{$\la{\neg b}{s}\rightarrow_B X'$}
    \end{prooftree}
    donde $X'$ es el booleano correspondiente (negación de $X$).


\item[]
\begin{prooftree}
    \AxiomC{$\la{b_1}{s}\rightarrow_B X$}
    \AxiomC{$\la{b_2}{s}\rightarrow_B Y$}
    \LeftLabel{}
    \RightLabel{}
    \BinaryInfC{$\la{b_1 \land b_2}{s}\rightarrow_B Z$}
    \end{prooftree}
donde $Z$ es el booleano correspondiente (conjunción de $X$ e $Y$).   
\end{itemize}
\end{sist*}
\end{example}
El siguiente resultado es análogo al último que dimos antes:
\begin{prop}
Sean $b \in \mathbf{Bexp}$, $s \in \mathbf{State}$ y $X \in Bool$. Entonces $\la{b}{s}\rightarrow_B X$ si y solo si $\fc{B}{b}s = X$.
\end{prop}
\begin{proof}
Por inducción estructural. Véase la demostración del anterior teorema y la definición de $\fc{B}{\cdot}$. La demostración es análoga.
\end{proof}

\section{Semántica operacional estructural}

\subsection{Sistema de transiciones}

Ahora nos centramos en los pasos concretos de la ejecución de un programa. Para ello, definimos una relación de transición $\la{S}{s} \dto \gamma$ como:
\begin{itemize}
    \item Si $\gamma$ es de la forma $\la{S'}{s'}$, entonces la ejecución de $S$ desde $s$ no se completa y sigue en $\la{S'}{s'}$.
    \item Si $\gamma$ es de la forma $s'$, entonces la ejecución finaliza en el estado $s'$.
\end{itemize}
La nueva relación de transición queda determinada por el conjunto de reglas:

\begin{sist*}[$\nn{While}_\nn{sos}$]\mbox{}
\begin{itemize}
    \item[] $[\text{ass}_{\nn{sos}}]$
\begin{prooftree}
    \AxiomC{}
    \LeftLabel{}
    \UnaryInfC{ $\la{x:=a}{s} \dto s[x\mapsto \fc{A}{a}s]$}
\end{prooftree}
    \item[]$[\text{skip}_{\nn{sos}}]$
\begin{prooftree}
    \AxiomC{}
    \LeftLabel{}
    \UnaryInfC{$\la{\n{skip}}{s} \dto s$}
\end{prooftree}
    \item[]$[\text{comp}_{\nn{sos}}^1]$
\begin{prooftree}
    \AxiomC{$\la{S_1}{s} \dto \la{S_1'}{s'}$}
    \LeftLabel{}
    \UnaryInfC{ $\la{S_1; S_2}{s} \dto \la{S_1'; S_2}{s'}$}
\end{prooftree}
    \item[]$[\text{comp}_{\nn{sos}}^2]$
\begin{prooftree}
    \AxiomC{$\la{S_1}{s} \dto s'$}
    \LeftLabel{}
    \UnaryInfC{ $\la{S_1; S_2}{s} \dto \la{S_2}{s'}$}
\end{prooftree}
    \item[]$[\text{if}^{\nn{tt}}_{\nn{sos}}]$
\begin{prooftree}
    \AxiomC{}
    \LeftLabel{}
    \RightLabel{si $\fc{B}{b}s = \textbf{tt}$}
    \UnaryInfC{$\la{\n{if}\ b\ \n{then}\ S_1\ \n{else}\ S_2}{s} \dto \la{S_1}{s}$}
\end{prooftree}
    \item[]$[\text{if}_{\nn{sos}}^\nn{ff}]$
\begin{prooftree}
    \AxiomC{}
    \LeftLabel{}
    \RightLabel{si $\fc{B}{b}s = \textbf{ff}$}
    \UnaryInfC{$\la{\n{if}\ b\ \n{then}\ S_1\ \n{else}\ S_2; s}{s} \dto \la{S_2}{s}$}
\end{prooftree}

    \item[] $[\nn{while}_\nn{sos}]$
\begin{prooftree}
    \AxiomC{}
    \LeftLabel{}
    \RightLabel{}
    \UnaryInfC{$\la{\n{while }b\n{ do }S}{s}\dto \la{\n{if }b\n{ then }(S; \n{while }b\n{ do }S)\n{ else skip}}{s}$}
\end{prooftree}    
\end{itemize}
\end{sist*}
Notemos que podríamos haber incluido, por ejemplo, dos reglas para la semántica de $\n{while }b \n{ do }S$:
\begin{prooftree}
    \AxiomC{}
    \LeftLabel{$[\nn{while}_\nn{sos}^\nn{ff}]$}
    \RightLabel{ si $\fc{B}{s}= \mathbf{ff}$}
    \UnaryInfC{$\la{\n{while }b\n{ do }S}{s}\dto s$}
\end{prooftree}    
y
\begin{prooftree}
    \AxiomC{}
    \LeftLabel{$[\nn{while}_\nn{sos}^\nn{tt}]$}
    \RightLabel{ si $\fc{B}{s}= \mathbf{tt}$}
    \UnaryInfC{$\la{\n{while }b\n{ do }S}{s}\dto \la{S;\n{while }b\n{ do }S}{s}$}
\end{prooftree}    
\begin{definition}
Se dirá que $\la{S}{s}$ está \textit{bloqueada} si no existe $\gamma$ tal que $\la{S}{s} \dto \gamma$. Una secuencia de derivación es finita cuando llega a un bloqueo o a un estado final:
\[
    \gamma_0 \dto \gamma_1 \dto ... \dto \gamma_k
\]
donde $\gamma_0 = \la{S}{s}$,  $\gamma_i \dto \gamma_{i+1}$ para $i\in\{0, ..., k-1\}$ y $\gamma_k$ es una configuración bloqueada.
\end{definition}

Normalmente escribiremos $\gamma_0 \dto^i \gamma$ si hay $i$ pasos en la ejecución de $\gamma_0$ a $\gamma$. Si hay finitos pasos, denotamos $\gamma_0 \dto^* \gamma$. $\gamma_0 \dto^i \gamma$ y $\gamma_0 \dto^* \gamma$ no tiene por qué ser secuencias de derivación, solo si $\gamma$ es configuración final o de bloqueo.

\begin{definition}
La ejecución $\la{S}{s}$ de la expresión $S$ en un estado $s$:
\begin{enumerate}
    \item \textit{Termina} si existe una única secuencia de derivación finita comenzando en $\la{S}{s}$.

    \item \textit{Termina con éxito} si $\la{S_1}{s} \dto ^* s'$ para algún estado $s'$.
    
    \item \textit{Cicla} si existe una secuencia de derivación infinita comenzando en $\la{S}{s}$.
\end{enumerate}
Nótese que estas definiciones son mutuamente excluyentes si y solo si las secuencias de derivacion son únicas. Por comodidad, las definimos de este modo porque, si extendemos el lenguaje, no nos tendremos que preocupar.
\end{definition}

\begin{example}\label{examplerepeat}
Supongamos que queremos extender $\nn{While}_\nn{sos}$ con la expresión $\n{repeat }S \n{ until }b$. Podemos añadir la regla:
\begin{prooftree}
        \AxiomC{}
        \LeftLabel{$[\nn{repeat}_\nn{sos}]$}
        \RightLabel{}
        \UnaryInfC{$\la{\n{repeat }S\n{ until }b}{s}\dto \la{S; \n{if }b\n{ then skip else }(\n{repeat }S\n{ until }b)}{s}$}
\end{prooftree}
La idea es que la expresión $\n{repeat }S\n{ until }b$ sea equivalente a $S;\n{ while }\neg b\n{ do } S$. Se definirá posteriormente el concepto de equivalencia semántica y se demostrará este resultado.
\end{example}

\subsection{Propiedades}

El método de demostración principal consiste en hacer \textit{inducción sobre la longitud de las secuencias de derivación} (finitas) que se estudian, es decir, si queremos demostrar una propiedad acerca de nuestro sistema de transiciones:
\begin{itemize}
    \item Demostramos que la propiedad se cumple para secuencias de derivación de longitud 0 (en ocasiones nos encontraremos que se cumple la propiedad de forma vacía).
    \item Demostramos que si la propiedad se cumple para secuencias de longitud (a lo sumo) $k$, entonces se cumple para secuencias de longitud $k+1$.
\end{itemize}
A modo de ejemplo de este método, veamos el siguiente resultado:

\begin{lema}\label{lemasos1}
Si $\la{S_1;S_2}{s}\dto^k s''$, entonces existen $s'\in \mathbf{State}$, $k_1, k_2 \in \N$ tales que $k = k_1 + k_2$ y $$\la{S_1}{s}\dto^{k_1}s' \quad \text{ y }\quad \la{S_2}{s'}\dto^{k_2}s''.$$
\end{lema}
\begin{proof}
Si $k=0$, entonces $\la{S_1;S_2}{s}\dto^0 s''$ implica (vacuamente) el resultado, porque $\la{S_1;S_2}{s}$ y $s''$ son distintos. Supongamos que el resultado se cumple para longitudes menores o iguales que $k$. Veamos que se sigue para $k+1$. Por tanto, tenemos la premisa $\la{S_1;S_2}{s}\dto^{k+1} s''$, es decir, que existe una configuración $\gamma$ tal que
$$\la{S_1;S_2}{s}\dto \gamma \dto^{k} s''$$
Por tanto, distinguimos dos casos según la regla que hemos aplicado a $\la{S_1;S_2}{s}$ para llegar a $\gamma$:
\begin{itemize}
    \item[(a)] Si hemos aplicado [$\nn{comp}^1_\nn{sos}$], tenemos que
\begin{prooftree}
    \AxiomC{$\la{S_1}{s} \dto \la{S_1'}{s'}$}
    \LeftLabel{$[\text{comp}_{\nn{sos}}^1]$}
    \UnaryInfC{ $\la{S_1; S_2}{s} \dto \la{S_1'; S_2}{s'}= \gamma$}
\end{prooftree}
luego $\la{S_1'; S_2}{s'} \dto^k s''$. Entonces, como esta derivación es de longitud $k$, podemos aplicar hipótesis de inducción, esto es, existen $s_0 \in \mathbf{State}$ y $k_1, k_2 \in \N$ con $k = k_1 + k_2$ y 
$$\la{S_1'}{s'}\dto^{k_1}s_0 \quad \text{ y }\quad \la{S_2}{s_0}\dto^{k_2}s''.$$
Ahora bien, como tenemos la premisa $\la{S_1}{s} \dto \la{S_1'}{s'}$ y $\la{S_1'}{s'}\dto^{k_1}s_0$, entonces tenemos que $\la{S_1}{s}\dto^{k_1 + 1}s_0$. Por otro lado, también tenemos $\la{S_2}{s_0}\dto^{k_2}s''$ y que $(k_1 + 1) + k_2 = (k_1 + k_2)+1 = k+1$. Es decir, hemos obtenido la conclusión deseada. Por tanto, hemos probado el resultado para este caso.

\item[(b)] Si hemos aplicado [$\nn{comp}^2_\nn{sos}$], tenemos que
\begin{prooftree}
    \AxiomC{$\la{S_1}{s} \dto s'$}
    \LeftLabel{[$\nn{comp}^2_\nn{sos}$]}
    \UnaryInfC{ $\la{S_1; S_2}{s} \dto \la{S_2}{s'}= \gamma$}
\end{prooftree}
Entonces deducimos que $\la{S_2}{s'} \dto^{k}s''$. Simplemente tomando $k_1 := 1$ y $k_2 := k$ vemos que $k_1+k_2 = k+1$ y que tenemos el resultado.
\end{itemize}
\end{proof}

\begin{example}
Por otro lado, $\la{S_1;S_2}{s}\dto^* \la{S_2}{s'}$ no implica necesariamente que $\la{S_1}{s}\dto^* s'$. Por ejemplo, podemos tomar $S_1 := \n{skip}$, $S_2 := \n{while }\neg(\n{x}=1)\n{ do } \n{x}:=\n{x}+1$ y $s\n{x} = 3$, $s'\n{x} = s[\n{x}\mapsto 2]$.
\end{example}


El siguiente lema viene a decir que la ejecución de una expresión es independiente de cualquier enunciado que se ejecute después:

\begin{lema}\label{lemasos2}
Si $\la{S_1}{s}\dto^k s'$, entonces $\la{S_1;S_2}{s}\dto^k \la{S_2}{s'}$.
\end{lema}
\begin{proof}
Por inducción sobre la longitud de las derivaciones. En caso de $k=0$, la premisa es falsa y el resultado se tiene directamente. Supongamos que se cumple el resultado para longitudes $\leq k$ y veámoslo para $k+1$. Nuestra suposición es que $\la{S_1}{s}\dto^{k+1} s'$. Entonces tenemos que hay cierta configuración $\gamma$ con 
$$\la{S_1}{s}\dto \gamma \dto^k s'$$
y además, notemos que $\gamma = \la{S}{s''}$ porque $k\leq 1$. Pero entonces, aplicando la hipótesis de inducción a $\la{S}{s''} \dto^k s'$, tenemos que $\la{S;S_2}{s''}\dto^k \la{S_2}{s'}$. 

Por otro lado, de $\la{S_1}{s}\dto \la{S}{s''}$ podemos deducir que:
\begin{prooftree}
    \AxiomC{$\la{S_1}{s} \dto \la{S}{s''}$}
    \LeftLabel{$[\text{comp}_{\nn{sos}}^1]$}
    \UnaryInfC{ $\la{S_1; S_2}{s} \dto \la{S; S_2}{s''}$}
\end{prooftree}
Es decir, sabemos que $\la{S_1; S_2}{s} \dto \la{S; S_2}{s''}$ y que $\la{S;S_2}{s''}\dto^k \la{S_2}{s'}$. Basta componer ambas derivaciones para ver que $\la{S_1; S_2}{s}\dto^{k+1} \la{S_2}{s'}$, como queríamos.
\end{proof}

\begin{theorem}\label{teosos}
El sistema de transiciones $\nn{While}_\nn{sos}$ es determinista, es decir, para cualesquiera $S, s, \gamma, \gamma'$ tenemos que
$$\la{S}{s}\dto \gamma \text{ y } \la{S}{s}\dto \gamma' \text{ implica que } \gamma = \gamma'$$
\end{theorem}
\begin{proof}
Véase la demostración del Teorema \ref{determinismo}.
\end{proof}

\begin{definition}
Dos expresiones $S_1, S_2$ se dicen \textit{semánticamente equivalentes} si, para cada $s \in \mathbf{State}$, 
\begin{itemize}
    \item Si $\gamma$ es estado final o bloqueado, entonces $\la{S_1}{s}\dto^* \gamma$ si y solo si $\la{S_2}{s} \dto^*\gamma$. Nótese que las longitudes de las derivaciones no tienen por qué coincidir.
    \item La\footnote{La unicidad viene dada por el determinismo de $\nn{While}_\nn{sos}$.} secuencia de derivación empezando en $\la{S_1}{s}$ es infinita si y solo si lo es la que empieza en $\la{S_2}{s}$.
\end{itemize} 
\end{definition}

\begin{example}
Veamos que $S$ y $S; \n{skip}$ son semánticamente equivalentes. Supongamos que $\la{S}{s}\dto^*s'$, es decir, existe $k\in\N$ tal que $\la{S}{s}\dto^*s'$. Entonces, el Lema \ref{lemasos2} nos dice que dado $\la{S}{s}\dto^k s'$ se tiene
\[
    \la{S; \n{skip}}{s}\dto^k \la{ \n{skip}}{s'}
\]
Por la regla $[\nn{skip}_\nn{sos}]$ se deduce $\la{ \n{skip}}{s'} \dto s'$ y entonces
\[
    \la{S;  \n{skip}}{s} \dto^* s'
\]
Supongamos ahora que $\la{S; \n{skip}}{s}\dto^*s''$, entonces existe $k\in\N$ tal que  $\la{S; \n{skip}}{s}\dto^k s''$. Por el Lema \ref{lemasos1} se tiene la existencia de $s'\in\textbf{State}$ y $k_1,k_2\in\N$ tal que $k = k_1 + k_2$ y
\[
    \la{S}{s}\dto^{k_1} s'\ \ y\ \ \la{\n{skip}}{s'}\dto^{k_2} s''
\]
La secuencia $\la{\n{skip}}{s'}\dto^{k_2} s''$ es cierta si y solo si $k_2 = 1$ y $s' = s''$ pues solo se puede aplicar $[\nn{skip}_\nn{sos}]$, se deduce entonces
\[
    \la{S}{s}\dto^{k_1} s''
\]
es decir, $\la{S}{s}\dto^{*} s''$.
\end{example}

\begin{example} Definimos en \ref{examplerepeat} la sentencia $\n{repeat }S\n{ until }b$ y dejamos por hacer la demostración de su equivalencia con $S;\n{ while }\neg b\n{ do } S$. Hay que probar que para cada $s, s'\in\textbf{State}$ sucede que
\[
    \la{\n{repeat }S\n{ until }b}{s} \dto^* s' \Longleftrightarrow \la{S;\n{ while }\neg b\n{ do } S}{s} \dto^* s'
\]
$\Longrightarrow)$ Supongamos que  
\[
    \la{\n{repeat }S\n{ until }b}{s} \dto^* s'
\]
Sabemos por la regla $[\nn{repeat}_\nn{sos}]$
\[
    \la{\n{repeat }S\n{ until }b}{s} \dto^* s'  \implies \la{S;\n{ if }b\n{ then skip else }(\n{repeat }S\n{ until }b)}{s} \dto^* s'
\]
y por el lema \ref{lemasos1} existen $k_1, k_2\in\N$ tal que
\[
    \la{S}{s} \dto^{k_1} s''
\]
y
\[
    \la{\n{if }b\n{ then skip else }(\n{repeat }S\n{ until }b)}{s''} \dto^{k_2} s'
\]
Dividimos en casos en función del valor booleano de $b$:
\begin{itemize}
    \item $\fc{B}{b}s'' = \textbf{tt}$. Usando el axioma $[\nn{if}^\nn{tt}_{sos}]$ seguido de $[\nn{skip}_\nn{sos}]$ se tiene
    \begin{eqnarray*}
        \la{\n{if }b\n{ then skip else }(\n{repeat }S\n{ until }b)}{s''} &\overset{[\nn{if}^\nn{tt}_{sos}]}{\dto}& \la{\n{skip}}{s''} \\
        &\overset{[\nn{skip}_\nn{sos}]}{\dto}& s''
    \end{eqnarray*}
    de lo que se deduce, por determinismo, que  $s' = s''$.
    Desarrollemos ahora $\la{S;\n{ while }\neg b\n{ do } S}{s}$. Por el lema \ref{lemasos2}, como sabemos que $\la{S}{s} \dto^{k_1} s''$, deducimos que
    \[
        \la{S;\n{ while }\neg b\n{ do } S}{s} \dto^{k_1} \la{\n{while }\neg b\n{ do } S}{s''}
    \]
    Aplicando $[\nn{while}_{sos}]$ deducimos $\la{\n{if }\neg b\n{ then } S;\n{ while }\neg b\n{ do }S\n{ else skip}}{s''}$ y como $\fc{B}{\neg b} = \textbf{ff}$ entonces
    \begin{eqnarray*}
            \la{\n{if }\neg b\n{ then } S;\n{ while }\neg b\n{ do }S\n{ else skip}}{s''} \dto &\overset{[\nn{if}^\nn{ff}_{sos}]}{\implies}& \la{\n{skip}}{s''} \\
            &\overset{[\nn{skip}_\nn{sos}]}{\implies}& s''
    \end{eqnarray*}
    concluyendo que $\la{S;\n{ while }\neg b\n{ do } S}{s} \dto^* s'$, como queríamos probar.
    \item $\fc{B}{b}s'' = \textbf{ff}$. Aplicando $[\nn{if}_\nn{ff}]$ se tiene
    \begin{eqnarray*}
        \la{\n{if }b\n{ then skip else }(\n{repeat }S\n{ until }b)}{s''} &\overset{[\nn{if}^\nn{tt}_{sos}]}{\dto}& \la{\n{repeat }S\n{ until }b}{s''}
    \end{eqnarray*}
    y esta última sentencia $\dto^* s'$, por determinismo. Por hipótesis de inducción se deduce de $$\la{\n{repeat }S\n{ until }b}{s''} \dto^* s'$$ que
    \[
        \la{S;\n{ while }\neg b\n{ do } S}{s''} \dto^* s'
    \]
    y por el lema \ref{lemasos2} y aplicando $[\nn{while}_\nn{sos}]$
    \begin{eqnarray*}
        \la{S;\n{ while }\neg b\n{ do } S}{s}  \dto^* \la{\n{ while }\neg b\n{ do } S}{s''} \dto \la{\n{if }\neg b\n{ then } S;\n{ while }\neg b\n{ do }S\n{ else skip}}{s''}
    \end{eqnarray*}
    Dado que $\fc{B}{\neg b}s'' = \textbf{tt}$ se tiene 
    \[
        \la{\n{if }\neg b\n{ then } S;\n{ while }\neg b\n{ do }S\n{ else skip}}{s''} \dto \la{S;\n{ while }\neg b\n{ do } S}{s''} \dto^* s'
    \]
    cocluyendo el resultado $\la{S;\n{ while }\neg b\n{ do } S}{s} \dto^* s'$.
\end{itemize}
$\Longleftarrow)$ Suponemos $\la{S;\n{ while }\neg b\n{ do } S}{s} \dto^* s'$. Nuevamente por el lema \ref{lemasos1} se tienen $k_1, k_2\in\N$ y $s''\in\State$ tal que
\[
    \la{S}{s} \dto^{k_1} s''\ \ \ y\ \ \ \la{\n{while }\neg b\n{ do } S}{s''} \dto^{k_2} s'
\]
De $[\nn{while}_\nn{sos}]$ se tiene
\[
    \la{\n{while }\neg b\n{ do } S}{s''} \dto \la{\n{if }\neg b\n{ then } S;\n{ while }\neg b\n{ do }S\n{ else skip}}{s''}
\]
Se divide en casos según $\fc{B}{b}s''$:
\begin{itemize}
    \item Si $\fc{B}{b}s'' = \textbf{tt}$ ($\fc{B}{\neg b}s'' = \textbf{ff}$) entonces se aplica $[\nn{if}_\nn{sos}^\nn{ff}]$ y $[\nn{skip}_\nn{sos}]$ para deducir 
    \[
        \la{\n{if }\neg b\n{ then } S;\n{ while }\neg b\n{ do }S\n{ else skip}}{s''} \dto \la{\n{skip}}{s''} \dto s''
    \]
    y como $\la{\n{while }\neg b\n{ do } S}{s''} \dto^k_1 s'$, por determinismo debe suceder $s' = s''$. Veamos que la sentencia del $\n{repeat}$ lleva al estado $s'$. Aplicamos la regla $[\nn{repeat}_\nn{sos}]$ para obtener
    \[
        \la{\n{repeat }S\n{ until }b}{s} \dto \la{S; \n{if }b\n{ then skip else }(\n{repeat }S\n{ until }b)}{s}
    \]
    Ahora como $\la{S}{s} \dto^{k_1} s''$ aplicando el lema \ref{lemasos2} se consigue que 
    \[
        \la{S;\n{if }b\n{ then skip else }(\n{repeat }S\n{ until }b)}{s} \dto^{k_1} \la{\n{if }b\n{ then skip else }(\n{repeat }S\n{ until }b)}{s''}
    \]
    Y ahora como $\fc{B}{b}s'' = \textbf{tt}$ se consigue mediante $[\nn{if}_\nn{sos}^\nn{tt}]$ y $[\nn{skip}_\nn{sos}]$
    \[
        \la{\n{if }b\n{ then skip else }(\n{repeat }S\n{ until }b)}{s''} \dto \la{\n{skip}}{s''} \dto s''
    \]
    Dado que $s' = s''$ se consigue finalmente
    \[
         \la{\n{repeat }S\n{ until }b}{s} \dto^* s'
    \]
    \item Si $\fc{B}{b}s'' = \textbf{ff}$ ($\fc{B}{\neg b}s'' = \textbf{tt}$) entonces toca aplicar la regla $[\nn{if}_\nn{sos}^\nn{tt}]$
    \begin{eqnarray*}
        \la{\n{if }\neg b\n{ then } S;\n{ while }\neg b\n{ do }S\n{ else skip}}{s''} &\dto& \la{S; \n{while }\neg b\n{ do }S}{s''} \\
        &\dto^*& s'
    \end{eqnarray*}
    Donde en el último paso se ha aplicado determinismo.
    Volvemos a aplicar lo mismo de antes para el $\n{repeat}$
    \begin{eqnarray*}
        \la{\n{repeat }S\n{ until }b}{s} &\overset{[\nn{repeat}_\nn{sos}]}{\dto}& \la{S; \n{if }b\n{ then skip else }(\n{repeat }S\n{ until }b)}{s} \\
        &\overset{\ref{lemasos2}}{\dto}& \la{\n{if }b\n{ then skip else }(\n{repeat }S\n{ until }b)}{s''} \\
        &\overset{[\nn{if}_\nn{sos}^\nn{ff}]}{\dto}& \la{\n{repeat }S\n{ until }b}{s''}
    \end{eqnarray*}
    Por hipótesis de inducción se tiene que
    \[
        \la{\n{repeat }S\n{ until }b}{s''} \dto^* s' \Longleftrightarrow \la{S; \n{while }\neg b\n{ do }S}{s''} \dto^* s'
    \]
    de donde se deduce que $\la{\n{repeat }S\n{ until }b}{s''} \dto^* s'$ y finalmente
    \[
         \la{\n{repeat }S\n{ until }b}{s} \dto^* s'
    \]
\end{itemize}
\end{example}
\subsection{Expresiones}

Análogamente a como hicimos en la semántica de paso largo, podemos definir el valor semántico de las expresiones mediante una función parcial $\mc{S}_\nn{sos}: \mathbf{Stm}\rightarrow (\mathbf{State}\hookrightarrow\mathbf{State})$, donde
\begin{align*}
    \mc{S}_\nn{sos}[[S]]:  \mathbf{State} & \hookrightarrow  \mathbf{State} \\
                     s               & \mapsto          \begin{cases} s' \text{, si } \la{S}{s}\dto^* s' \\ 
                        \n{indefinido} \text{, en otro caso}
                        \end{cases}
\end{align*}
Notemos que esta función está bien definida precisamente por el determinismo que vimos en el anterior apartado. La ejecución de $\la{S}{s}$ para una expresión $S$ dada una configuración inicial $s\in\textbf{State}$ puede dar lugar a tres casos
\begin{itemize}
    \item Que termine, y entonces existe $s'\in\textbf{State}$ tal que
    \[
        s\in\textbf{State} \dto^* s'
    \]
    y entonces $\mc{S}_\nn{sos}[[S]]s = s'$
    \item Que se quede bloqueada, y entonces no queda otra que $\mc{S}_\nn{sos}[[S]]s = \n{indefinido}$.
    \item Que cicle, ocurriendo nuevamente que $\mc{S}_\nn{sos}[[S]]s = \n{indefinido}$.
\end{itemize} 
La equivalencia semántica coincide $\mc{S}_\nn{sos}$ en el sentido de la siguiente proposición: \\

\begin{prop} Si $S_1$ y $S_2$ son semánticamente equivalente entonces
\[
        \mc{S}_\nn{sos}[[S_1]] = \mc{S}_\nn{sos}[[S_2]]
\] 
\begin{proof} Hay que probarlo particularizando en cada $s\in\textbf{State}$ \\ \\
Supongamos que $S_1$ y $S_2$ son semánticamente equivalentes. Se distinguen dos casos
\begin{itemize}
    \item Existe un estado final $s'\in\textbf{State}$ tal que
    \[
        \la{S_1}{s}\dto^* s'\ \ y\ \ \la{S_2}{s} \dto^* s'
    \]
    y trivialmente
    \[
        \mc{S}_\nn{sos}[[S_1]]s = s' =  \mc{S}_\nn{sos}[[S_2]]s
    \]
    \item Ambas secuencias son infinitas, es decir
    \[
        \la{S_1}{s}\text{ es infinita si y solo si } \la{S_2}{s} \text{ es infinita}
    \]
    y entonces
    \[
        \mc{S}_\nn{sos}[[S_1]]s = \n{indefinido} =  \mc{S}_\nn{sos}[[S_2]]s
    \]
\end{itemize}
\end{proof}
\end{prop}
\noindent La implicación contraria no es cierta, pues si $S_1$ es una sentencia que cicla y $S_2$ es una sentencia que se bloquea, $ \mc{S}_\nn{sos}[[S_1]] = \n{indefinido} = \mc{S}_\nn{sos}[[S_2]]$ pero $S_1$ y $S_2$ no son semánticamente equivalentes.

\section{Teorema de equivalencia}

Hasta ahora hemos presentado por separado los dos sistemas de transiciones para $\nn{While}$, y hemos visto que tienen comportamientos, en general, distintos. Sin embargo, si vemos la semántica de paso largo como una extensión de la de paso corto, podemos convencernos que son esencialmente los mismo. De hecho, tienen el mismo poder expresivo. Más precisamente, el resultado clave de esta sección es:

\begin{theorem}[De equivalencia]\label{theq}
Consideremos el lenguaje $\nn{While}$. Para cada $S \in \mathbf{Stm}$, $\mc{S}_\nn{ns}[[S]] = \mc{S}_\nn{sos}[[S]]$.
\end{theorem}
\begin{proof}
La demostración se divide en dos pasos, cada uno consiste en demostrar uno de los siguientes lemas:
\begin{lema}
Para cada $S \in \mathbf{Stm}$, dados $s, s' \in \mathbf{State}$, $\la{S}{s}\rightarrow s'$ implica que $\la{S}{s}\dto^* s'$.
\end{lema}
\begin{proof}
La demostración vuelve a ser rutinaria, es decir, por inducción en la forma del árbol de derivación. Los casos base son:
\begin{itemize}
    \item[($\nn{ass}_\nn{ns}$)] Si suponemos que $\la{x:=a}{s}\to s$, el axioma $[\nn{ass}_\nn{sos}]$ nos da el resultado.
    \item[($\nn{skip}_\nn{ns}$)] Análogo.
\end{itemize}
Los casos inductivos son:
\begin{itemize}
    \item[($\nn{comp}_\nn{ns}$)]  Supongamos que $\la{S_1;S_2}{s}\to s''$. Esto necesariamente se ha deducido de las premisas $\la{S_1}{s}\to s'$ y $\la{S_2}{s'}\to s''$ y, aplicando la hipótesis de inducción, $\la{S_1}{s}\dto^*s'$ y $\la{S_2}{s'}\dto^* s''$. Por el Lema \ref{lemasos2}, $\la{S_1;s_2}{s}\dto^* \la{S_2}{s'}\dto^* s''$.
    \item[($\nn{if}_\nn{ns}^\nn{tt}$)] Supongamos que hemos obtenido $\la{\n{if }b \n{ then }S_1\n{ else }S_2}{s}\to s'$ a partir de $\fc{B}{b}s = \mathbf{tt}$ y $\la{S_1}{s}\to s'$. Pero entonces, aplicando $[\nn{if}_\nn{sos}^\nn{tt}]$, $\la{\n{if }b \n{ then }S_1\n{ else }S_2}{s}\dto \la{S_1}{s}$ y, por la hipótesis de inducción en $\la{S_1}{s}\to s'$, $\la{S_1}{s}\dto^* s'$.
    \item[($\nn{if}_\nn{ns}^\nn{ff}$)] Análogo.
    \item[($\nn{while}_\nn{ns}^\nn{tt}$)] Supongamos que $\la{\n{while }b \n{ do }S}{s}\to s''$ se ha deducido de las premisas $\fc{B}{b}s = \mathbf{tt}$, $\la{S}{s}\to s'$ y $\la{\n{while }b \n{ do }S}{s'}\to s''$. La hipótesis de inducción en las dos últimas nos da que $\la{S}{s}\dto^* s'$ y que $\la{\n{while }b \n{ do }S}{s'}\dto^* s''$. Por el Lema \ref{lemasos2}, 
    $$\la{S_1;\n{while }b \n{ do }S}{s} \dto^* \la{\n{while }b \n{ do }S}{s'}\dto^* s''$$
    Juntando ésto y aplicando en orden $[\nn{while}_\nn{sos}^\nn{tt}]$ y $[\nn{if}_\nn{sos}^\nn{tt}]$, 
    $$\la{\n{while }b \n{ do }S}{s} \dto \la{\n{if }b\n{ then }(S; \n{while }b \n{ do }S) \n{ else skip}}{s} \dto \la{S;\n{while }b \n{ do }S}{s}\dto^* s''$$
    \item[($\nn{while}_\nn{ns}^\nn{ff}$)] Análogo.
\end{itemize}
\end{proof}

\begin{lema}
Para cada $S \in \mathbf{Stm}$, dados $s, s' \in \mathbf{State}$ y $k \in \N$, $\la{S}{s}\dto^k s'$ implica que $\la{S}{s}\to s'$.
\end{lema}
\begin{proof}
Se demuestra por inducción en la longitud de las secuencias de derivación. El caso base es $k=0$ y, como la premisa es falsa, la implicación es cierta directamente.
\\

Supongamos que el resultado se cumple si $k \leq k_0$ y veamos que es cierto para $k_0 +1$. Tenemos que distinguir casos en función de la regla que ha sido aplicada para obtener el primer paso de la derivación $\la{S}{s}\dto^{k_0+1}s'$.
\begin{itemize}
    \item[($\nn{ass}_\nn{sos}$)] Es evidente porque no requiere ninguna premisa. Además, $k_0 = 0$ (se obtiene la derivación en un paso).
    \item[($\nn{skip}_\nn{sos}$)] Análogo.
    \item[($\nn{comp}_\nn{sos}^{1, 2}$)] Supongamos que $\la{S_1;S_2}{s}\dto^{k_0+1}s''$. Por el Lema \ref{lemasos1}, existen un estado $s'$ y enteros $k_1 + k_2 = k_0 + 1$ tales que 
    $$\la{S_1}{s}\dto^{k_1}s' \quad \text{ y } \quad \la{S_2}{s'}\dto^{k_2}s''.$$
    Como $k_1, k_2 \leq k_0$, podemos aplicar la hipótesis de inducción en cada caso, y obtenemos que $\la{S_1}{s}\to s'$ y $\la{S_2}{s'}\to s''$. Aplicando $[\nn{comp}_\nn{ns}]$, obtenemos que $\la{S_1;S_2}{s}\to s''$.
    \item[($\nn{if}_\nn{sos}^\nn{tt}$)] Supongamos que el primer paso es de la forma:
    $$\la{\n{if }b\n{ then }S_1\n{ else }S_2}{s}\dto \la{S_1}{s} \dto^{k_0} s'$$
    es decir, con $\fc{B}{b}s = \mathbf{tt}$. Aplicando la hipótesis de inducción a la segunda secuencia, $\la{S_1}{s} \to s'$, luego basta aplicar $[\nn{if}_\nn{ns}^\nn{tt}]$ para tener el resultado.
    \item[($\nn{if}_\nn{sos}^\nn{ff}$)] Análogo.
    \item[($\nn{while}_\nn{sos}$)] Supongamos que tenemos:
    $$\la{\n{while }b \n{ do }S}{s}\dto \la{\n{if }b \n{ then }(S; \n{while }b\n{ do }S)\n{ else skip}}{s} \dto^{k_0}s''$$
    Aplicando la hipótesis de inducción al segundo paso, $\la{\n{if }b \n{ then }(S; \n{while }b\n{ do }S)\n{ else skip}}{s} \to s''$. Por el Lema \ref{lemans}, tenemos el resultado.
\end{itemize}
\end{proof}

Entonces, dados $S \in \mathbf{Stm}$, $s\in \mathbf{State}$, por los dos lemas se tiene que $\mc{S}_\nn{ns}[[S]] = s'$ si y solo si $\mc{S}_\nn{sos}[[S]] = s'$. Pero entonces obtenemos que, si una de las dos funciones está definida en un estado o no, la otra lo estará también y coincidirá con ella o no estará definida, respectivamente. Es decir, ambas funciones coinciden.
\end{proof}



\cleardoublepage
\chapter{Más semántica operacional}
						
$\\$


La elección de una semántica operacional u otra para el lenguaje $\nn{While}$ es, por el Teorema de equivalencia, arbitraria. En cambio, para otros lenguajes puede ocurrir que una aproximación sea natural y la otra completamente impracticable. En esta sección estudiaremos una serie de conceptos que mostrarán la debilidad de las semánticas que hemos visto hasta ahora. 

\section{Construcciones no secuenciales}
\subsection{\n{abort}}

La intención que tenemos al introducir la expresión $\n{abort}$ es la siguiente: queremos que, cuando se ejecute, pare la ejecución de todo el programa. Notemos que no podríamos, a simple vista, construir una expresión con el mismo comportamiento en $\nn{While}$. A saber, $\n{while true do skip}$ solo consigue hacer un bucle, y $\n{skip}$ permite que un programa se pueda ejecutar más tarde. Entonces, la sintaxis de las expresiones (o sentencias) queda del siguiente modo:
\[
    \begin{array}{l}
         S ::= x:=a\ |\ \n{skip}\ |\ S_1;S_2\ |\ \n{if}\ b\ \n{then}\ S_1\ \n{else}\ S_2\ |\ \n{while}\ b\ \n{do}\ S \ | \ \n{abort}
    \end{array}
\]
Por otro lado, anque debemos modificar ahora el comportamiento de las relaciones de transición que vimos en el anterior capítulo, no es necesario alterar los sistemas de transiciones asociados, porque tan solo queremos que cualquier configuración de la forma $\la{\n{abort}}{s}$ quede atascada, y esto no modifica el comportamiento de (las reglas asociadas a) las otras expresiones. Es decir, la expresión $\n{abort}$ no necesita de ninguna regla que determine su interpretación.

Ahora bien, no es cierto que $\nn{While}_\nn{ns}$ y $\nn{While}_\nn{sos}$ se comporten, en general, del mismo modo. De hecho, $\n{abort}$ distingue entre ambos sistemas de semántica operacional. Esto se debe a que en $\nn{While}_\nn{ns}$ solo nos interesan las ejecuciones que terminan correctamente y por tanto no distinguimos entre bucles o configuraciones atascadas, mientras que en $\nn{While}_\nn{sos}$ podemos definir bucles (secuencias de derivación infinitas) y ejecuciones que terminan incorrectamente (secuencias de derivación finitas que terminan en una configuración atascada). Así, aunque $\n{abort}$ no sea semánticamente equivalente a $\n{skip}$ en $\nn{While}_\nn{ns}$, sí que lo es $\n{while true do skip}$. Por otro lado, vemos que $\n{abort}$ no puede ser semánticamente equivalente en $\nn{While}_\nn{sos}$ a $\n{while true do skip}$, porque a partir de $\la{\n{while true do skip}}{s}$ hay una secuencia de derivación infinita y a partir de $\la{\n{abort}}{s}$ no. Tampoco puede serlo $\n{skip}$, porque $\la{\n{skip}}{s}\dto s$ es la única secuencia de derivación posible empezando en $\n{skip}$ y $\la{\n{abort}}{s}$ es la correspondiente a $\n{abort}$. 

\begin{example}
Podemos extender $\nn{While}$ con la expresión $\n{assert }b\n{ before }S$. La idea es que, si $b$ es cierto, entonces ejecutamos $S$ y, si es falso, entonces la ejecución del programa se aborta. [HACER ESTO]
\end{example}


\subsection{\n{or}}

Otra posible extensión de las expresiones de $\nn{While}$ consiste en añadir nuevas posibilidades de ejecución, es decir, forzar un tipo de no determinismo. Para ello incluimos la expresión $\n{or}$:
\[
    \begin{array}{l}
         S ::= x:=a\ |\ \n{skip}\ |\ S_1;S_2\ |\ \n{if}\ b\ \n{then}\ S_1\ \n{else}\ S_2\ |\ \n{while}\ b\ \n{do}\ S \ | \ S_1 \n{ or } S_2
    \end{array}
\]
Así, si por ejemplos ejecutamos $\n{x}:= 1 \n{ or } \n{x}:=2$, entonces generamos dos caminos distintos: uno en el que $\n{x}$ pasa a tener el valor 1 y otro en el que toma el valor 2. Como siempre, en caso de que sea preciso, emplearemos paréntesis para indicar de manera precisa las distintas opciones.


A diferencia de la expresión $\n{abort}$, debemos distinguir la nueva ampliación de $\nn{While}$ en función de cada semántica operacional:
\begin{sist*}[$\nn{While}_\nn{ns}$ con \n{op}]
A las reglas de $\nn{While}_\nn{ns}$ añadimos:
\begin{itemize}
    \item[]
\begin{prooftree}
    \AxiomC{$\la{S_1}{s}\to s'$}
    \LeftLabel{[$\nn{or}^1_\nn{ns}$]}
    \RightLabel{}
    \UnaryInfC{$\la{S_1 \n{ or } S_2}{s}\to s'$}
    \end{prooftree}

    \item[]
\begin{prooftree}
    \AxiomC{$\la{S_2}{s}\to s'$}
    \LeftLabel{[$\nn{or}^2_\nn{ns}$]}
    \RightLabel{}
    \UnaryInfC{$\la{S_1 \n{ or } S_2}{s}\to s'$}
    \end{prooftree}
\end{itemize}
\end{sist*}


\begin{sist*}[$\nn{While}_\nn{sos}$ con \n{op}]
A las reglas de $\nn{While}_\nn{sos}$ añadimos:
\begin{itemize}
    \item[]
\begin{prooftree}
    \AxiomC{}
    \LeftLabel{[$\nn{or}^1_\nn{sos}$]}
    \RightLabel{}
    \UnaryInfC{$\la{S_1 \n{ or } S_2}{s}\dto \la{S_1}{s}$}
    \end{prooftree}

    \item[]
\begin{prooftree}
    \AxiomC{}
    \LeftLabel{[$\nn{or}^2_\nn{sos}$]}
    \RightLabel{}
    \UnaryInfC{$\la{S_1 \n{ or } S_2}{s}\dto \la{S_2}{s}$}
    \end{prooftree}
\end{itemize}
\end{sist*}

De nuevo, como cabía esperar, ambos sistemas de transiciones funcionan de maneras distintas respecto del no determinismo que hemos introducido. En el caso de $\nn{While}_\nn{ns}$ tenemos que el no determinismo omite la posibilidad de que haya bucles infinitos, mientras que en $\nn{While}_\nn{sos}$ esto no ocurre. Para ilustrar esto vamos a recurrir al ejemplo de antes.

En $\nn{While}_\nn{ns}$, a la configuración $\la{\n{x}:= 1 \n{ or } \n{x}:=2}{s}$ corresponden dos árboles de derivación, cada uno asociado a las siguientes transiciones:
$$\la{\n{x}:= 1 \n{ or } \n{x}:=2}{s} \to s[\n{x}\mapsto 1]$$
$$\la{\n{x}:= 1 \n{ or } \n{x}:=2}{s} \to s[\n{x}\mapsto 2]$$
Además, si tuviésemos $\la{(\n{while true do skip}) \n{ or } \n{x}:=2}{s}$, entonces solo habría un árbol de derivación posible, a saber, el que no corresponde al bucle, es decir, 
$$\la{(\n{while true do skip}) \n{ or } \n{x}:=2}{s} \to s[\n{x}\mapsto 2]$$
En cambio, en $\nn{While}_\nn{sos}$, las secuencias de derivación para $\la{\n{x}:= 1 \n{ or } \n{x}:=2}{s}$ serían
$$\la{\n{x}:= 1 \n{ or } \n{x}:=2}{s} \dto^* s[\n{x}\mapsto 1]$$
$$\la{\n{x}:= 1 \n{ or } \n{x}:=2}{s} \dto^* s[\n{x}\mapsto 2]$$
y para $\la{(\n{while true do skip}) \n{ or } \n{x}:=2}{s}$, 
$$\la{(\n{while true do skip}) \n{ or } \n{x}:=2}{s} \dto^* s[\n{x}\mapsto 2]$$
$$\la{(\n{while true do skip}) \n{ or } \n{x}:=2}{s} \dto \la{\n{while true do skip}}{s} \dto\dots$$
de donde se deduce lo que dijimos arriba.


\subsection{\n{par}}

Una manera de profundizar en el no determinismo del que hablamos antes consiste en que, por ejemplo, se dé la posibilidad de ejecutar dos expresiones, pero no de forma excluyente, es decir, que ambas ejecuciones se puedan ejecutar intercalándose. Esta es la idea detrás de la expresión $\n{par}$:
\[
    \begin{array}{l}
         S ::= x:=a\ |\ \n{skip}\ |\ S_1;S_2\ |\ \n{if}\ b\ \n{then}\ S_1\ \n{else}\ S_2\ |\ \n{while}\ b\ \n{do}\ S \ | \ S_1 \n{ par } S_2
    \end{array}
\]
En un caso sencillo como $\n{x}:= 1 \n{ par } \n{x}:=2$, tenemos solo los resultados posibles 1 y 2, correspondientes a ejecutar primero $\n{x}:= 1$ y después $\n{x}:=2$ y viceversa. Si por ejemplo tuviésemos $\n{x}:= 1 \n{ par } (\n{x}:=2; \n{x}:\n{x}:=\n{x} + 2)$, entonces habría tres posibles resultados:
\begin{itemize}
    \item Si ejecutamos $\n{x}:=1$ y luego $\n{x}:=2; \n{x}:= \n{x}+2$, obtenemos $4$.
    \item Si ejecutamos $\n{x}:=2; \n{x}:= \n{x}+2$ y luego $\n{x}:=1$, obtenemos $1$.
    \item Si ejecutamos $\n{x}:=2$, después $\n{x}:=1$ y por último $\n{x}:= \n{x}+2$, obtenemos 3.
\end{itemize}

Veamos cómo queda el lenguaje $\nn{While}$ ampliado con esta nueva expresión. En el caso de $\nn{While}_\nn{sos}$ tenemos:
\begin{sist*}[$\nn{While}_\nn{sos}$ con \n{par}]
A las reglas de $\nn{While}_\nn{sos}$ añadimos:
\begin{itemize}
    \item[]
\begin{prooftree}
    \AxiomC{$\la{S_1}{s}\dto \la{S_1'}{s'}$}
    \LeftLabel{[$\nn{par}^1_\nn{sos}$]}
    \RightLabel{}
    \UnaryInfC{$\la{S_1 \n{ par } S_2}{s}\dto \la{S'_1 \n{ par } S_2}{s'}$}
    \end{prooftree}

    \item[]
\begin{prooftree}
    \AxiomC{$\la{S_1}{s}\dto s'$}
    \LeftLabel{[$\nn{par}^2_\nn{sos}$]}
    \RightLabel{}
    \UnaryInfC{$\la{S_1 \n{ par } S_2}{s}\dto \la{S_2}{s'}$}
    \end{prooftree}

   \item[]
\begin{prooftree}
    \AxiomC{$\la{S_2}{s}\dto \la{S_2'}{s'}$}
    \LeftLabel{[$\nn{par}^3_\nn{sos}$]}
    \RightLabel{}
    \UnaryInfC{$\la{S_1 \n{ par } S_2}{s}\dto \la{S_1 \n{ par } S'_2}{s'}$}
    \end{prooftree}
    \item[]
\begin{prooftree}
    \AxiomC{$\la{S_2}{s}\dto s'$}
    \LeftLabel{[$\nn{par}^4_\nn{sos}$]}
    \RightLabel{}
    \UnaryInfC{$\la{S_1 \n{ par } S_2}{s}\dto \la{S_1}{s'}$}
    \end{prooftree}
\end{itemize}
\end{sist*}
\cleardoublepage
\chapter{Implementación correcta}
Una pregunta a realizarse sería cómo implementar un lenguaje de programación para que funcione en un ordenador. En este capítulo se mostrará, \textit{grosso modo}, cómo realizar una implementación basándonos en cómo funciona Java. Para ello, estudiaremos los siguientes conceptos:
\begin{itemize}
    \item Una máquina abstracta con unas instrucciones básicas.
    \item Traducción de $\mathbf{Stm}$ a código de la máquina abstracta.
    \item Corrección.
    \item Bisimulación.
\end{itemize}

La idea es que definiremos el comportamiento semántico de las instrucciones de la máquina abstracta mediante semántica operacional y después estudiaremos traducciones que lleven elementos de $\nn{While}$ en secuencias de estas instrucciones. Con \textit{corrección} nos referimos a que, si traducimos una expresión en \textit{código} y lo ejecutamos, entonces obtendremos el mismo resultado que el que ya describimos mediante $\mc{S}_\nn{ns}$ o $\mc{S}_\nn{sos}$. Finalmente, veremos de pasada la bisimulación.

\section{Máquina abstracta}


\subsection{Configuraciones e introducciones}

Una \textit{configuración} de la máquina abstracta (MA) consta de:
\begin{itemize}
    \item Una secuencia de instrucciones a ejecutar (o código), $s$.
    \item Una pila de evaluación, $e$.
    \item Un almacén, $s$,
\end{itemize}
Denotaremos por $\mathbf{Code}$ a la categoría sintáctica de las \textit{secuencias de instrucciones}. La pila de evaluación servirá para evaluar expresiones aritméticas y booleanas. Formalmente será una lista de valores:
\[
    \textbf{Stack} := (\Z\cup T)^*
\]
De esta forma, una configuración será una terna $\conf{c}{e}{s} \in \textbf{Code}\times\textbf{Stack} \times \textbf{State}$. Para mayor simplicidad, se considerá el almacén de forma parecida a como consideramos los elementos de $\textbf{State}$ para almacenar variables.
\\

La gramática de las instrucciones de la máquina abstracta es de la siguiente forma: 
\begin{eqnarray*}
    inst &::=& \n{PUSH}-n\ |\ \n{ADD}\ |\ \n{MULT}\ |\ \n{SUB} \\
        &|& \n{TRUE}\ |\ \n{FALSE}\ |\ \n{EQ}\ |\ \n{NEG}\ |\ \n{AND}\\
        &|& \n{FETCH}-x\ |\ \n{STORE}-x\\
        &|& \n{NOOP}\ |\ \n{BRANCH}(c,c)\ |\ \n{LOOP}(c,c)\\
    c  &::=& \varepsilon\ |\ inst : c
\end{eqnarray*}
con $c \in \mathbf{Code}$.
\\


Se denomina \textit{configuración final} a las ternas de la forma $\langle \varepsilon, e, s\rangle$, es decir, con el código vacío. La idea va a ser que, a partir de una configuración no final, haya una transición que vaya eliminando código y modificando la pila de evaluación y el almacén.
\\

Pasemos a describir el sistema de semántica operacional asociado a la MA. Definimos una relación de transición:
\[
    \conf{c}{e}{s} \rhd \conf{c'}{e'}{s'}
\]
que significa lo siguiente: la configuración $\conf{c'}{e'}{s'}$ se obtiene de $\conf{c}{e}{s}$ \textit{en un paso de ejecución}. Nótese la similitud con la semántica operacional de paso corto. Esta relación viene determinada por el siguiente sistema de transiciones:

\begin{sist*}[Semántica operacional para la MA]

\begin{eqnarray*}
    \conf{\n{PUSH}-n:c}{e}{s} &\rhd& \conf{c}{\fc{N}{n}:e}{s} \\
    \conf{\n{ADD}:c}{z_1:z_2:e}{s} &\rhd& \conf{c}{(z_1\oplus z_2):e}{s} \quad si \ z_1, z_2 \in \mathbf{Z} \\
    \conf{\n{MULT}:c}{z_1:z_2:e}{s} &\rhd& \conf{c}{(z_1\otimes z_2):e}{s} \quad si \ z_1, z_2 \in \mathbf{Z} \\
    \conf{\n{SUB}:c}{z_1:z_2:e}{s} &\rhd& \conf{c}{(z_1\ominus z_2):e}{s} \quad si \ z_1, z_2 \in \mathbf{Z} \\
    \conf{\n{TRUE}:c}{e}{s} &\rhd& \conf{c}{\textbf{tt}:e}{s} \\
    \conf{\n{FALSE}:c}{e}{s} &\rhd& \conf{c}{\textbf{ff}:e}{s} \\
    \conf{\n{EQ}:c}{z_1:z_2:e}{s} &\rhd& \conf{c}{(z_1 \eq z_2):e}{s} \quad si \ z_1, z_2 \in \mathbf{Z}\\
    \conf{\n{LE}:c}{z_1:z_2:e}{s} &\rhd& \conf{c}{(z_1 \leq z_2):e}{s}\quad si \ z_1, z_2 \in \mathbf{Z} \\
    \conf{\n{AND}:c}{t_1:t_2:e}{s} &\rhd& 
        \left\{\begin{array}{ll} 
            \conf{c}{\textbf{tt}:e}{s} & \text{si }t_1 = t_2 = \textbf{tt} \\
            \conf{c}{\textbf{ff}:e}{s} & \text{caso contrario}
        \end{array}\right. 
    \\
    \conf{\n{NEG}:c}{t:e}{s} &\rhd& 
        \left\{\begin{array}{ll} 
            \conf{c}{\textbf{ff}:e}{s} & \text{si }t = \textbf{tt} \\
            \conf{c}{\textbf{tt}:e}{s} & \text{si }t = \textbf{ff}
        \end{array}\right. \\
    \conf{\n{FETCH-x}:c}{x}{e} &\rhd&  \conf{c}{s(x):e}{s}
    \\
    \conf{\n{STORE-x}:c}{z:e}{s} &\rhd&  \conf{c}{e}{s[x\to z]} \quad si \ z \in \mathbf{Z}
    \\
    \conf{\n{NOOP}:c}{e}{s} &\rhd&  \conf{c}{e}{s} 
    \\
    \conf{\n{BRANCH}(c_1, c_2):c}{t:e}{s} &\rhd& 
        \left\{\begin{array}{ll} 
            \conf{c_1:c}{e}{s} & \text{si }t = \textbf{tt} \\
            \conf{c_2:c}{e}{s} & \text{si }t = \textbf{ff}
        \end{array}\right.
    \\
    \conf{\n{LOOP}(c_1,c_2):c}{e}{s} &\rhd&  \conf{c_1 : \n{BRANCH}(c_2, \n{LOOP}(c_1,c_2), \n{NOOP}:c}{e}{s} 
\end{eqnarray*}
\end{sist*}
Donde hemos usado `:' con doble sentido: por un lado, nos referimos a la concatenación de dos secuencias de isntrucciones (algo similar a `;' para $\mathbf{Stm}$) y, por otro, a la operación de anteponer un elemento a una tal secuencia. También conviene aclarar algunos de los códigos. El código $\n{LOOP}(c_1,c_2)$ es el más complejo y es el que servirá más adelante para traducir la sentencia $\n{While}$. El código $\n{BRANCH}(c_1, c_2)$ toma la idea de actuar como un $\n{if}$, siendo la condición la que esté en la cima de la pila.
\\ 

\subsection{Propiedades semánticas}

Como ya hemos dicho, el sistema de transiciones anterior no es más que un sistema de semántica operacional estructural. Por tanto, podemos reescribir los conceptos que ya vimos anteriomente para esta semántica. Por ejemplo, las secuencias de derivación se denominan \textit{secuencias de cómputo}. Una secuencia de cómputo para un código $c$ y un almacén $s$ puede ser \textit{finita}, es decir, de la forma
\[
    \gamma_0 \rhd \gamma_1 \rhd ... \rhd \gamma_k
\]
siendo $\gamma_i \in \textbf{Code}\times\textbf{Stack} \times \textbf{State}$ para todo $i\leq k$ y cumpliendo $\gamma_i \rhd \gamma_{i+1}$ para $i<k$. Por otro lado, denotamos una secuencia de cómputo \textit{infinita} como
\[
    \gamma_0 \rhd \gamma_1 \rhd ...
\]
con  $\gamma_i \rhd \gamma_{i+1}$ para $i\geq 0$. En ambos casos, la configuración inicial será de la forma $\gamma_0 := \conf{c}{\varepsilon}{s}$, es decir, las secuencias empiezan con la pila vacía. Además, decimos que una secuencia de cómputo:
\begin{itemize}
    \item \textit{Termina} si y solo si es finita.
    \item \textit{Cicla} si y solo si es infinita.
\end{itemize}
Nótese que, en el primer caso, se puede terminar con una configuración final o con una \textit{configuración atascada}, como por ejemplo $\langle \n{ADD}, \varepsilon, s\rangle$.
\\

Para no repetir las demostraciones que ya hemos visto, enunciaremos las diferentes propiedades básicas del sistema que acabamos de presentar:

\begin{lema}
Si $\conf{c_1}{e_1}{s}\rhd^k \conf{c'}{e'}{s'}$, entonces $\conf{c_1:c_2}{e_1:e_2}{s}\rhd^k \conf{c':c_2}{e':e_2}{s'}$.
\end{lema}
\begin{proof}
Véase la demostración del Lema \ref{lemasos2}.
\end{proof}

\begin{lema}
Si $\conf{c_1:c_2}{e}{s} \rhd^k \conf{\varepsilon}{e''}{s''}$, entonces existe una configuración $\conf{\varepsilon}{e'}{s'}$ y $k_1, k_2 \in \N$ tales que $k = k_1 + k_2$ y 
$$\conf{c_1}{e}{s} \rhd^{k_1} \conf{\varepsilon}{e'}{s'} \quad\text{ y }\quad \conf{c_2}{e'}{s'} \rhd^{k_2} \conf{\varepsilon}{e''}{s''}.$$
\end{lema}
\begin{proof}
Véase la demostración del Lema \ref{lemasos1}.
\end{proof}

\begin{theorem}
El sistema de transiciones para el código de la máquina abstracta es determinista, es decir, 
$$\gamma \rhd \gamma'  \text{ y }\gamma \rhd \gamma'' \quad \text{ implica } \quad \gamma' = \gamma''$$
\end{theorem}
\begin{proof}
Véase la demostración del Teorema \ref{teosos}.
\end{proof}

\subsection{Función de ejecución}

De nuevo, por analogía sobre lo que ya comentamos en el capítulo de semántica operacional estructural, aparece la idea de una función que  permita definir el valor semántico de cada expresión (aquí, secuencia de instrucciones). Más concretamente, definimos la función parcial $\mc{M}: \mathbf{Code} \rightarrow (\mathbf{State} \hookrightarrow \mathbf{State})$:
$$\fc{M}{c}s := \begin{cases}s' \text{, si } \conf{c}{\varepsilon}{s}\rhd^* \conf{\varepsilon}{e}{s'}\\ \n{INDEFINIDO} \text{, en otro caso.}\end{cases}$$
Que $\mc{M}$ esté bien definida se sigue por determinismo. Conviene hacer notar que esta definición requiere que la componente del código, en la configuración final, sea vacía.

\section{Traducción}
\cleardoublepage
\chapter{Semántica denotacional}


Hasta ahora, lo que nos ha interesado de los programas era su ejecución. En cambio, como ya dijimos en la introducción, lo que queremos estudiar ahora es el efecto de esta ejecución, es decir, la relación entre los estados iniciales y los finales asociados a ella.  Para ello, lo que haremos es definir una función semántica para cada categoría sintáctica, en el sentido de que cada construcción sintáctica será interpretada mediante un objeto matemático, el `efecto' de ejecutar tal construcción. 
\\

Este enfoque no es del todo desconocido para nosotros. Por ejemplo, las funciones semánticas $\mc{A}$ y $\mc{B}$ que ya hemos empleado son un ejemplo típico de definición denotacional: a cada expresión aritmética asociamos un objeto abstracto, más concretamente, una función de tipo $\mathbf{State}\rightarrow \mathbf{Z}$. Y lo mismo ocurre con las expresiones booleanas y las correspondientes funciones de $\mathbf{State}\rightarrow \mathbf{Bool}$. Recordemos que una característica importante de esta definiciones era la composicionalidad (véase la introducción). Por tanto, las funciones $\mc{S}_\nn{ns}$ y $\mc{S}_\nn{sos}$ no se corresponden con la idea de definición denotacional que hemos descrito.

\section{Sistema para While}

\subsection{Función semántica}

Las sentencias modifican el estado en el que se encuentran. Por tanto, en el enfoque denotacional, se define una aplicación que, dada una sentencia, devuelve una función (parcial) que transforma estados en estados:
\[
    \mc{S}_\nn{ds} : \mathbf{Stm} \to (\State \hookrightarrow \State)
\]
Notemos que $ \mc{S}_\nn{ds}$ es parcial precisamente por la sentencia $\n{while true do skip}$. Veamos, caso por caso, qué función es exactamente la que asignamos.
\\

Si reflexionamos en lo que implica la ejecución de $x:= a$, parecerá natural que le corresponda una aplicación que modifica, en cada estado, el valor que toma $x$. Más precisamente, una definición de este tipo sería:
\[
    \mc{S}_\nn{ds} [[x:=a]] : s \mapsto s[x\mapsto \fc{A}{a}s]
\]
es decir, dado un estado $s$, lo convertiremos a un estado en el cual la variable $x$ toma el valor $\fc{A}{a}s$. 
\\

En el caso de $\nskip$ queremos que no se ejecute ningún cambio en el estado. Lo reflejamos mediante la identidad $id : \State \to \State$, dada por $id(s) := s$\footnote{Se hace notar que podríamos dar definiciones de las aplicaciones recurriendo a imágenes de estados arbitrarios. En este caso no se ha utilizado la variable estado $s$, porque no la necesitamos para modificar el estado posterior (en el caso anterior necesitamos $s$ para saber el valor de la expresión aritmética $a$). Se podría haber definido, de forma equivalente, $\mc{S}_\nn{ds} [[\nskip]]s := s$, para todo $s \in \mathbf{State}$.}:
\[
    \mc{S}_\nn{ds} [[\nskip]] := id
\]

Para la composición de dos sentencias, $S_1; S_2$, simplemente queremos recibir el estado al que se llega tras ejecutar $S_1$ y después aplicar $S_2$. Esto lo podemos hacer mediante la composición:
\[
    \mc{S}_\nn{ds} [[S_1;S_2]] := \mc{S}_\nn{ds} [[S2]] \circ \mc{S}_\nn{ds} [[S_1]]
\]
Es muy importante tener en cuenta que $\mc{S}_\nn{ds} [[S_1]]$ se evalúa antes, pasando su resultado a $\mc{S}_\nn{ds} [[S_2]]$. 
\\

La idea para el condicional es la siguiente:
\[
    \mc{S}_\nn{ds} [[\nif b\nthen S_1\nelse S_2]]s := 
    \begin{cases}  
        \mc{S}_\nn{ds} [[S1]]s\text{, si } \fc{B}{b}s = \textbf{tt} \\
        \mc{S}_\nn{ds} [[S2]]s\text{, si } \fc{B}{b}s = \textbf{ff}
   \end{cases}
\]
donde sencillamente se ha utilizado el valor $\fc{B}{b}s$ para separar en dos casos y, según el resultado, aplicar una sentencia u otra. Pero, como dijimos antes, suele ser más cómodo emplear notación funcional. Podemos definir entonces una aplicación auxiliar $\n{cond} : (\State \to \mathbf{Bool}) \times (\State \hookrightarrow  \State) \times (\State \hookrightarrow  \State) \to (\State \hookrightarrow  \State)$, bajo la idea de que el primer parámetro será la evaluación de la expresión booleana (en función del estado) y las otras dos serán las evaluaciones de las sentencias $S_1$ y $S_2$\footnote{Hemos empleado el producto de tipos para mayor simplicidad. También podríamos haber escrito $\n{cond} : (\State \to \mathbf{Bool}) \to (\State \hookrightarrow  \State) \to (\State \hookrightarrow  \State) \to (\State \hookrightarrow  \State)$ y por tanto prescindir de paréntesis. Puede que, en un aspecto puramente formal, esta diferencia no sea relevante. En cambio, sí que es importante señalarla desde el punto de vista de los tipos de datos que emplea $\n{cond}$. También podríamos haber sustituido el tipo de funciones $\mathbf{State}\to \mathbf{Bool}$ por el tipo simple $\mathbf{Bool}$ pero, de nuevo, recordemos que es preferible trabajar con funciones.}. Para ello, según el valor que retorne el booleano, se devolverá el primer parámetro o el segundo, es decir:
\[
    \n{cond}(f_b, f_1, f_2) s := 
    \begin{cases}
        
        f_1s\text{, si } f_bs = \textbf{tt}  \\
        f_2 s\text{, si } f_bs = \textbf{ff} 
    \end{cases}
\]
Por tanto, la definición deseada es:
\[
    \mc{S}_\nn{ds} [[\nif b\nthen S_1\nelse S_2]] := \n{cond}(\fc{B}{b}, \mc{S}_\nn{ds} [[S_1]], \mc{S}_\nn{ds} [[S_2]])
\]
Esto es, si $b$ es cierto en el estado actual, se toma la primera sentencia y, en caso contrario, la segunda. 
\\

Hemos visto que ha sido posible definir las expresiones básicas mediante aplicaciones relativamente sencillas. No ocurrirá lo mismo con la sentencia $\nwhile b \ndo S$. Podríamos intentar usar lo anterior, junto con la equivaencia semántica con la sentencia $\n{if }b\n{ then }(S; \n{while }b\n{ do }S)\n{ else skip}$, y así obtendríamos:
$$\mc{S}_\nn{ds}[[\nwhile b \ndo S]] := \n{cond}(\fc{B}{b}, \mc{S}_\nn{ds}[[\n{while }b\n{ do }S]]\circ \mc{S}_\nn{ds}[[S]], id)$$
Pero, ¿acaso es esto una definición aceptable? Entre otras cosas, no es composicional, y por tanto no podemos aceptarla. La igualdad anterior sí que significa, por otro lado, algo, a saber: que $\mc{S}_\nn{ds}[[\nwhile b \ndo S]]$ es \textit{punto fijo} de la función $F$ dada por $F(g) := \n{cond}(\fc{B}{b}, g\circ \mc{S}_\nn{ds}[[S]], id)$. De hecho, $F$ sí que admite una definición composicional porque en lo anterior solo aparecen construcciones estructuralmente más sencillas que $\nwhile b \ndo S$. Por tanto, la siguiente definición es más satisfactoria:
$$\mc{S}_\nn{ds}[[\nwhile b \ndo S]] := \n{fix}(F)$$
donde $\n{fix}: ((\mathbf{State}\hookrightarrow \mathbf{State})\to (\mathbf{State}\hookrightarrow \mathbf{State}))\to (\mathbf{State}\hookrightarrow \mathbf{State})$ nos lleva una aplicación (de apliaciones) a su punto fijo. 

\begin{example}
Sea $\n{while } \neg(\n{x} = 0)\n{ do skip}$. El funcional asociado es, en este caso:
$$G(g)s := \begin{cases}
    gs \text{, si } s\n{x} \neq 0\\
    s \text{, en otro caso}
\end{cases}$$
¿Hay algún punto fijo para $G$? Probemos con 
$$g_1s := \begin{cases}
    \n{INDEFINIDO} \text{, si } s\n{x}\neq 0 \\
    s \text{, en otro caso}
\end{cases}$$
En efecto, 
$$G(g_1)s = \begin{cases}
    g_1s \text{, si } s\n{x} \neq 0\\
    s \text{, en otro caso}
\end{cases} = \begin{cases}
    \n{INDEFINIDO} \text{, si } s\n{x}\neq 0 \\
    s \text{, en otro caso}
\end{cases} = g_1s$$
Luego $g_1 = \n{fix}(G)$. Por otro lado, $g_2 \equiv \n{INDEFINIDO}$ no puede ser punto fijo de $G$ porque, si $s \in \mathbf{State}$ verifica $sx = 0$, entonces $G(g_2)s = s \neq \n{INDEFINIDO} = g_2s$.
\end{example}


Entonces, de momento, tenemos las siguientes reglas: 

\begin{sist*}[$\nn{While}_\nn{ds}$]\label{whileds}
\begin{eqnarray*}
    \mc{S}_\nn{ds} [[x:=a]]s &:=& s[x\mapsto \fc{A}{a}s] \\
    \mc{S}_\nn{ds} [[\nskip]] &:=& id \\
    \mc{S}_\nn{ds} [[S_1;S_2]] &:=& \mc{S}_\nn{ds} [[S2]] \circ \mc{S}_\nn{ds} [[S_1]] \\
    \mc{S}_\nn{ds} [[\nif b\nthen S_1\nelse S_2]]s &=&  \n{cond}(\fc{B}{b}, \mc{S}_\nn{ds} [[S_1]], \mc{S}_\nn{ds} [[S_2]])\\
    \mc{S}_\nn{ds} [[\nwhile b \ndo S]] &:=&  \n{fix}(F)\\
\end{eqnarray*}
donde $F(g) := \n{cond}(\fc{B}{b}, g\circ \mc{S}_\nn{ds}[[S]], id)$.
\end{sist*}

Notemos que podríamos haber intentado definir 
$$\mc{S}_{ds}[[\n{while } b \n{ do } S]]s := \mc{S}_{ds}^{\delta(b, S)s}s$$
con 
$$\delta(b, S)s := \begin{cases}
    0 \text{, si } \fc{B}{b}s = \mathbf{ff} \\
    1 + \delta(b, S)s' \text{, en otro caso}
\end{cases}$$
Sería una definición composicional pero, en cambio, la aplicación $\delta(b, S)$ no está bien definida como función recursiva, porque podría darse el caso de que no devolviera ningún valor bien definido.

\subsection{Requisitos de punto fijo}

Ahora bien, si examinamos la definición que hemos dado para $\mc{S}_\nn{ds} [[\nwhile b \ndo S]]$, veremos que volvemos a tener problemas. Esta vez el obstáculo se encuentra en que hemos supuesto que cada aplicación asociada a una sentencia de la forma $\nwhile b \ndo S$ tiene \textit{un} punto fijo. La realidad es bien distinta: puede haber varios puntos fijos o incluso ninguno, como ilustran los siguientes ejemplos.

\begin{example}
Cualquier elemento $f \in  \mathbf{State}\hookrightarrow \mathbf{State}$, tal que $fs =s$ si $sx = 0$, es punto fijo de la aplicación $G$ del anterior ejemplo.
\end{example}

\begin{example}
Sean $g_1, g_2 \in \mathbf{State}\hookrightarrow\mathbf{State}$. La aplicación 
$$H(g) := \begin{cases}
     g_1 \text{, si } g = g_2\\
     g_2 \text{, en otro caso}
\end{cases}$$
no admite ningún punto fijo si $g_1 \neq g_2$ porque, si $H(g) = g$, entonces, o bien $g \neq g_2$ y $H(g)= g_2$, o bien $g = g_2$ y $H(g)= g_2 \neq g_1=g$.
\end{example}

Por tanto, como no podemos admitir que $\n{fix}$ devuelva un conjunto de candidatos a puntos fijos, parece irremediable dar una serie de requerimientos para seleccionar \textit{el} que nos interesa. La tarea, entonces, es doble, porque debemos probar que al menos y a lo sumo uno de los posibles puntos fijos verifica tales condiciones.
¿Qué condiciones dar? No entraremos mucho en detalle aquí, pero conviene tener una idea general de por qué exigimos determinadas restricciones. Podemos comenzar estudiando las diferentes configuraciones que se obtienen al ejecutar $\nwhile b \ndo S$ desde un estado $s_0$. Sea $F$ la aplicación asociada. Tenemos los siguientes casos:
\begin{itemize}
    \item[(a)] La ejecución termina. Por ejemplo, con $\n{while } 0≤\n{x do x} := \n{x}−1$ desde cualquier estado en el que $\n{x}\geq 0$.
    \item[(b)] La ejecución entra en un bucle local, es decir, hay alguna subexpresión en $S$ que entra en bucle. Por ejemplo, en $\n{while }0≤\n{x do }(\n{if x}=0 \n{ then }(\n{while true do skip})\n{ else x}:= \n{x}−1)$ desde cualquier estado con $\n{x}\geq 0$.
    \item[(c)] La ejecución entra en un bucle global, es decir, se entra en bucle al ejecutar la expresión completa. Por ejemplo, con $\n{while } ¬(\n{x}=0)\n{ do skip}$ desde cualquier estado con $\n{x}\neq 0$.
\end{itemize}
Más concretamente, 
\begin{itemize}
    \item[(a)] Hay una secuencia de estados $s_1, \dots, s_n$ tales que $\fc{B}{b}$ toma el valor $\mathbf{ff}$ únicamente en $s_n$ y con $\mc{S}_\nn{ds}[[S]]s_i=s_{i+1}$ para $i <n$. Si $F(g) = g$, entonces
    $$gs_i = F(g)s_i = \n{cond}(\fc{B}{b}, g\circ \mc{S}_\nn{ds}[[S]], id)$$
    y entonces es claro que $gs_i = s_{i+1}$, si $i <n$, y $gs_i = s_i$ si $i = n$. Pero entonces $gs_0 = s_n$.
    \item[(b)] Hay una secuencia de estados $s_1, \dots, s_n$ tales que $\fc{B}{b}$ toma el valor $\mathbf{tt}$ para cada $i \leq n$ y tales que $\mc{S}_\nn{ds}[[S]]s_i = s_{i+1}$, si $i <n$, y $\mc{S}_\nn{ds}[[S]]s_i = \n{INDEFINIDO}$, si $i = n$. Por un razonamiento análogo al anterior, podemos determinar que todo $g = F(g)$ verifica que $gs_0 = \n{INDEFINIDO}$.
    \item[(c)] Existe una secuencia infinita de estados $s_1, s_2, \dots$ tales que $\fc{B}{b}s_i =\mathbf{tt}$ y $\mc{S}_\nn{ds}[[S]]s_i = s_{i+1}$, para cada $i=1, 2, \dots$. Entonces, es fácil ver que, si $F(g)=g$,
$$gs_{i} = gs_{i+1}, \text{ para todo } i \in \N$$
Pero entonces $gs_0 = gs_i$, para cada $i\in \N$. Por tanto, siguiendo este método no podemos determinar el valor de $gs_0$.
\end{itemize}

Precisamente, esta anomalía del caso (c) es lo que nos indica que los posibles puntos fijos de $F$ pueden diferir entre sí. Para mayor claridad, centrémonos en el siguiente ejemplo:

\begin{example}
Consideremos $\n{while } ¬(\n{x}=0)\n{ do skip}$, ejecutado desde un estado $s_0$ con $\n{x}\neq 0$. En este caso, 
$$F(g)s := \begin{cases}
    gs \text{, si } s\n{x} \neq 0\\
    s \text{, en otro caso}
\end{cases}$$
Pero ya vimos que cualquier $g \in \mathbf{State}\hookrightarrow\mathbf{State}$ con $gs = s$ si $s\n{x} = 0$ era punto fijo de $F$. Por otro lado, está claro que la intuición nos dice que sería deseable tener
$$\mc{S}_\nn{ds}[[\n{while } ¬(\n{x}=0)\n{ do skip}]]s_0 := \begin{cases}
    \n{INDEFINIDO} \text{, si } s_0\n{x} = 0 \\
    s_0 \text{, en otro caso}
\end{cases}$$
Pero entonces, si tomamos ahora
$$hs := \begin{cases}
    \n{INDEFINIDO} \text{, si } s\n{x} = 0 \\
    s \text{, en otro caso}
\end{cases}$$
es claro que $F(h) = h$ y, \textit{además}, que $hs = s'$ implica $h's = s'$, para cualquier otro punto fijo $h'$ de $F$, pero no al contrario. 
\end{example}
Este ejemplo muestra que, en resumen, el candidato a $\n{fix}(F)$ debe ser cierto $h \in \mathbf{State}\hookrightarrow\mathbf{State}$ tal que:
\begin{itemize}
    \item[(1)] $F(h)=h$.
    \item[(2)] Sean $s, s' \in \mathbf{State}$. Si $g \in \mathbf{State}\hookrightarrow\mathbf{State}$ verifica (1), entonces
    $$hs =s' \text{ implica que } gs=s'$$
\end{itemize}
Nótese que, si $hs = \n{INDEFINIDO}$ entonces no se verifica (2). 


\begin{example}[Ejercicio 5.2] El funcional F asociado con la sentencia
\[
    \nwhile \neg(x=0) \ndo x:= x-1
\]
es, por definición $F(g) = \n{cond}(\fc{B}{\neg(x=0)}, g\circ\mathcal{S}_\nn{ds}[[x:= x-1]], \n{id})$, que aplicado a un estado $s$ puede verse de la forma
\begin{eqnarray*}
    F\ g (s) &=& \left\{\begin{array}{ll}
          g\circ\mathcal{S}_\nn{ds}[[x:= x-1]], \n{id}) (s) & \fc{B}{\neg(x=0)}s = \textbf{tt} \\
          s &  c.c
    \end{array}\right. \\
    &=& \left\{\begin{array}{ll}
          g(s[x \mapsto (s\ x) - 1]) & s\ x\neq 0 \\
          s &  s\ x = 0
    \end{array}\right.
\end{eqnarray*}
Ahora vamos a ver varios ejemplos de función $g$ y comprobaremos si son puntos fijos por $F$:
\begin{enumerate}
    \item La aplicación
    \[
        g_1\ s := \n{INDEFINIDO},\ \ \forall s\in\State
    \]
    no es un punto fijo. Tomemos por ejemplo un estado $s\in\State$ tal que $s\ x = 1$, entonces
    \begin{eqnarray*}
        s' &:=& (F\ g_1)\  s \\
        &=& \left\{\begin{array}{ll}
        g_1\ s[x \mapsto 0] & x\neq 0 \\
          s &  c.c\end{array}\right. \\
        &=& g_1\ s[x \mapsto 0]
    \end{eqnarray*}
    luego $s'\ x = \n{INDEFINIDO} \neq 1 = s\ x$, por lo que $g$ no es un punto fijo.
    \item La aplicación
    \[
         g_2\ s = \left\{\begin{array}{ll}
          s[x\mapsto 0] & s\ x \geq 0 \\
          \n{INDEFINIDO} &  s\ x < 0
    \end{array}\right.
    \]
    sí es un punto fijo. Sea $s\in\State$, supongamos $s\ x > 0$:
    \begin{eqnarray*}
        s' &:=& (F\ g_2)\  s \\
        &=& \left\{\begin{array}{ll}
        g_2\ s[x \mapsto s\ x - 1] & s\ x\neq 0 \\
          s &  s\ x=0\end{array}\right. \\
        &=& g_2\ s[x \mapsto s\ x - 1] \\
        &=& s[[x \mapsto s\ x - 1]][x \mapsto 0] \\
        &=& s[x \mapsto 0] = g_2\ s
    \end{eqnarray*}
    Para el caso $s\ x = 0$ sucede que
    \[
        (F\ g_2)\  s = s = s[x\mapsto 0] = g_2\ s
    \]
    Y finalmente si $x<0$, informalmente el bucle sería infinito:
    \[
        (F\ g_2)\  s = \n{INDEFINIDO} = g_2\ s
    \]
    Por lo que se deduce que $F\ g_2 = g_2$, luego es un punto fijo.
    \item Consideremos la aplicación
     \[
         g_3\ s = \left\{\begin{array}{ll}
          s[x\mapsto 0] & s\ x \geq 0 \\
          s &  s\ x < 0
    \end{array}\right.
    \]
    Tomemos $s\ x = -1$ entonces
    \begin{eqnarray*}
        (F\ g_3)\  s  &=&  g_3\ s[x \mapsto -2] \\
        &=& g_3\ s[x \mapsto -2] \neq s = g_3\ s
    \end{eqnarray*}
    Por lo que $F\ g_3 \neq g_3$ y entonces $g_3$ no es punto fijo.
    \item Definimos ahora $g_4\ s = s[x\mapsto 0]$ para todo $s\in\State$. Esta función es punto fijo:
    \begin{eqnarray*}
        (F\ g_4)\  s  &=& \left\{\begin{array}{ll}
        g_4\ s[x \mapsto s\ x -1] & s\ x\neq 0 \\
          s &  s\ x=0\end{array}\right. \\
        &=& \left\{\begin{array}{ll}
        s[x \mapsto 0] & x\neq 0 \\
        s &  s\ x = 0 \end{array}\right. \\
        &=& s[x \mapsto 0] = g_4\ s
    \end{eqnarray*}
    \item La aplicación identidad $g_5\ s = s$ no es un punto fijo. Sea $s\in\State$ tal que $s\ x = 1$ entonces
    \begin{eqnarray*}
        (F\ g_5)\ s  &=& g_5\ s[x\mapsto 0] \neq s = g_5\ s
    \end{eqnarray*}
    de lo que se deduce que $F\ g_5 \neq g_5$.
\end{enumerate}
\end{example}

\begin{example}[Ejercicio 5.3] Consideramos ahora la sentencia
\[
    \nwhile \neg(x=1)\ndo (y:=y*x; x:= x-1)
\]
Tiene por funcional $F$:
\begin{eqnarray*}
    (F\ g) s &=& \left\{\begin{array}{ll}
          g\circ\mathcal{S}_\nn{ds}[[y:=y*x; x:= x-1]], \n{id}) (s) & \fc{B}{\neg(x=1)}s = \textbf{tt} \\
          s &  c.c
    \end{array}\right. \\
    &=& \left\{\begin{array}{ll}
          g\ s[y\mapsto (s\ y)*(s\ x)][x\mapsto (s\ x)-1]& s\ x\neq 1 \\
          s &  s\ x = 1
    \end{array}\right.
\end{eqnarray*}
Este bucle hace un cálculo de $y\cdot x!$, por lo que es previsible que el punto fijo represente este hecho. \\ \\
Un ejemplo de punto fijo es la aplicación
\[
    f_1\ s = \left\{\begin{array}{ll}
    s[y\mapsto (s\ y)*(s\ x)!][x\mapsto 1] & si\ s\ x\geq 1 \\
    \n{INDEFINIDO} & \text{caso contrario}
    \end{array}\right.
\]
pues
\begin{eqnarray*}
    (F\ f_1)\ s &=& \left\{\begin{array}{ll}
          f_1\ s[y\mapsto (s\ y)*(s\ x)][x\mapsto (s\ x)-1]& s\ x\neq 1 \\
          s &  s\ x = 1
    \end{array}\right. \\
    &=& \left\{\begin{array}{ll}
          f_1\ s'[y\mapsto (s'\ y)*(s'\ x)][x\mapsto 1] & s\ x\neq 1 \land s'\geq 1 \\
          \n{INDEFINIDO} & s\ x\neq 1 \land s'< 1 \\
          s &  s\ x = 1
    \end{array}\right. \\
    &=& \left\{\begin{array}{ll}
          f_1\ s[y\mapsto (s\ y)*(s\ x)!][x\mapsto 1] & s\ x \geq 2 \\
          \n{INDEFINIDO} & s\ x < 1 \\
          s &  s\ x = 1
    \end{array}\right.\\ &=& f_1\ s
\end{eqnarray*}
donde $s' := s[y\mapsto (s\ y)*(s\ x)][x\mapsto (s\ x)-1]$-
También es un punto fijo
\[
    f_2\ s = s[y\mapsto (s\ y)*(s\ x)!][x\mapsto 1]
\]
pues es igual a $f_1$ salvo en los puntos en los que la primera función está indefinida.
\end{example}

\section{Conjuntos parcialmente ordenados}

Antes de formalizar todas las ideas anteriores, conviene repasar algunos resultados básicos sobre conjuntos parcialmente ordenados. Posteriormente, haremos referencia a éstos, aplicados a un caso especial que describiremos más adelante.

\begin{definition}
Un par $(D,  \sqsubseteq)$ formado por un conjunto $D$ y una relación $\sqsubseteq \ \subseteq D^2$ se dice \textit{conjunto parcialmente ordenado} si $\sqsubseteq$ es reflexiva, antisimétrica y transitiva. 
\end{definition}

Recordemos también que un elemento $d \in D$ se dice $\sqsubseteq$-\textit{mínimo} si, para cada $d' \in D$, $d \sqsubseteq d'$.

\begin{lema}
Si un conjunto parcialmente ordenado $(D, \sqsubseteq)$ tiene un elemento mínimo, entonces es único. 
\end{lema}
\begin{proof}
En otro caso, si hubiera dos mínimos, bastaría aplicar la definición de elemento mínimo y la antisimetría del orden parcial para comprobar que son iguales.
\end{proof}

Por tanto, si existe, denotaremos a tal elemento mínimo como $\bot$.

\begin{definition} 
Dados $(D, \sqsubseteq)$ parcialmente ordenado y $Y\subset D$, se dice que $d\in D$ es una \textit{cota superior} de $Y$ si, para cada $d' \in Y$, $d' \sqsubseteq d$.
\end{definition}

Llamaremos \textit{supremo} de $Y \subseteq D$ a la menor cota superior de $Y$ (respecto de $\sqsubseteq$).

\begin{lema}
Si $Y \subseteq D$ tiene un supremo respecto de un orden parcial, entonces es único.
\end{lema}
\begin{proof}
Basta repetir la demostración del lema anterior, esta vez para cotas superiores.
\end{proof}

Si existe tal supremo en $Y$, lo denotamos por $\bigsqcup Y$. Pasemos ahora a un concepto que será relevante más adelante. Recordemos que un conjunto $A$ se dice \textit{totalmente ordenado} por una relación de orden $<$ si, dados arbitrarios $x, y \in A$, $x < y$ o $y < x$.

\begin{definition}
Un subconjunto $C$ de un conjunto parcialmente ordenado $(D, \sqsubseteq)$ se llama \textit{cadena} si es totalmente ordenado.
\end{definition}

\begin{definition}
Un conjunto parcialmente ordenado $(D, \sqsubseteq)$ se llama \textit{completo} si, dada una cadena cualquiera $C$, existe $\bigsqcup C$. Si lo mismo ocurre para todos los subconjuntos de $D$, lo denominamos \textit{retículo completo}.
\end{definition}

\begin{prop}
Sea $(D, \sqsubseteq)$ es un conjunto parcialmente ordenado y completo. Entonces tiene un elemento mínimo $\bot := \bigsqcup \emptyset$.
\end{prop}
\begin{proof}
Claramente, $\emptyset$ es cadena (verifica la afirmación de forma vacía) y, como $D$ es completo por hipótesis, existe $\bigsqcup \emptyset$. Ahora bien, dado $d \in D$, por la definición de $\bigsqcup \emptyset$ es directo que se verifica $\bigsqcup\emptyset \sqsubseteq d$, luego obtenemos el resultado.
\end{proof}

\begin{definition}
Sean $(D, \leq)$ y $(E, \preccurlyeq)$ conjuntos ordenados. Una aplicación $f: D \to E$ se dice \textit{monótona} si $d \leq d'$ implica que $f(d) \preccurlyeq f(d')$, para cualesquiera $d, d' \in D$. 
\end{definition}

Es directo comprobar que la composición de aplicaciones monótonas es monótona.

\begin{lema}
Sean $(D, \sqsubseteq), (E, \preccurlyeq)$ conjuntos parcialmente ordenados completos y $f: D \to E$ monótona. Entonces, si $C \subseteq D$ es cadena, $f(C) \subseteq E$ también lo es. Además\footnote{Nótese que, en la expresión que aparece en el teorema, nos referimos, primero a un $\preccurlyeq$-supremo, y después a un $\sqsubseteq$-supremo. Confiamos en el lector para que esté atento al uso en cada caso.},
$$\bigsqcup f(C) \preccurlyeq f\left(\bigsqcup C\right).$$
\end{lema}
\begin{proof}
Si $C = \emptyset$, entonces $f(\emptyset)= \emptyset$ y, como $\bot_E\preccurlyeq f(\bot_D)$, tenemos el resultado. Por tanto, supongamos que $C \neq \emptyset$. Sean $e_1, e_2 \in f(C)$. Entonces existen $d_1, d_2 \in C$ con $f(d_1)=e_1$ y $f(d_2)=e_2$. Por ser $C$ cadena, $d_1 \sqsubseteq d_2$ o $d_2 \sqsubseteq d_1$ y, por monotonía de $f$, $e_1 \preccurlyeq e_2$ o $e_2 \preccurlyeq e_1$, luego $f(C)$ es efectivamente cadena.
\\

Para lo segundo, sea $d \in C$. Entonces, por definición, $d \sqsubseteq \bigsqcup C$ y, por monotonía, $f(d) \preccurlyeq f(\bigsqcup C)$. Es decir, $f(\bigsqcup C)$ es cota superior de $f(C)$. Pero entonces, por definición, $\bigsqcup f(C) \preccurlyeq f(\bigsqcup C)$.
\end{proof}

Aunque, en general, no es cierto que las aplicaciones monótonas preserven supremos, nos centraremos en ellas a continuación:
\begin{definition}
Sean $(D, \sqsubseteq), (E, \preccurlyeq)$ conjuntos parcialmente ordenados completos y $f: D \to E$ monótona. Decimos que $f$ es \textit{continua} si $\bigsqcup f(C) = f(\bigsqcup C)$ para cada cadena $C \neq \emptyset$. Si, además, $f(\bot_D) = \bot_E$, decimos que $f$ es \textit{estricta}.
\end{definition}

Es fácil comprobar que, si $f: D \to E$ es continua, entonces es monótona. También es fácil ver que composición de funciones continuas es continua, y que lo mismo ocurre para las funciones estrictas. 
\\

Llegamos de este modo al concepto que ha motivado toda la sección:

\begin{theorem}\label{fixth}
Sea $(D, \sqsubseteq)$ conjunto parcialmente ordenado completo. Sea $f: D \to D$ continua. Entonces,
$$\text{\n{fix}}(f) := \bigsqcup\{f^m(\bot) \, | \, m \in \N\}$$
define un elemento de $D$ y es precisamente el menor punto fijo de $f$\footnote{Aquí empleamos la notación $f^0 := id$, $f^{n+1} := f^n\circ f$.}.
\end{theorem}
\begin{proof}
Veamos que $\n{fix}(f)$ está bien definido. Primero, $id(\bot) = \bot$ y, claramente, $\bot \sqsubseteq d$ para cada $d \in D$. Supongamos que $f^n(\bot) \sqsubseteq f^n(d)$ para cada $d \in D$ y cierto $n > 0$. Entonces, por monotonía, $f^{n+1}(\bot) = f(f^n{\bot}) \sqsubseteq f(f^n(d)) = f^{n+1}(d)$. Entonces, es fácil comprobar que $f^n(\bot) \sqsubseteq f^{m}(\bot)$ si $n \leq m$. Pero entonces, $C:= \{f^n(\bot) \, | \, n \in \N\}$ es cadena no vacía en $D$ y, por completitud, existe $\n{fix}(f)$.
\\

Veamos que $f(\n{fix}(f)) = \n{fix}(f)$:
$$f(\n{fix}(f)) = f\left(\bigsqcup\{f^n(\bot) \, | \, n \geq 0\}\right) = \bigsqcup\{f^{n+1}(\bot) \, | \, n \geq 0\} = \bigsqcup\{f^{n}(\bot) \, | \, n \geq 1\}$$ 
porque $f$ es continua. Ahora, usando que $\bigsqcup (C \cup \{\bot\}) = \bigsqcup C$, para cada cadena $C$, y que $\bot = f^0(\bot)$,
$$\bigsqcup\{f^{n}(\bot) \, | \, n \geq 1\} = \bigsqcup(\{f^{n}(\bot) \, | \, n \geq 1\}\cup \{\bot\}) = \bigsqcup\{f^{n}(\bot) \, | \, n \geq 0\} = \n{fix}(f).$$
Finalmente, veamos que es el menor punto fijo. Sea $d$ otro punto fijo de $f$. Como $\bot \sqsubseteq d$, por monotonía se tiene que $f^n(\bot) \sqsubseteq f^n(d)$, para cada $n \geq 0$. Como $d$ es punto fijo, $f^n(\bot) \sqsubseteq d$, para cada $n \geq 0$, es decir, $d$ es cota superior de la cadena $\{f^n(d) \, | \, n \geq 0\}$. Pero, por definición, $\n{fix}(f) \sqsubseteq d$.
\end{proof}

\begin{prop}
Sea $f: D \to D$ continua con $(D, \sqsubseteq)$ parcialmente ordenado y completo. Sea $d \in D$ tal que $f(d)\sqsubseteq d$. Entonces, $\n{fix}(f) \sqsubseteq d$.
\end{prop}
\begin{proof}
Para empezar, por monotonía, deducimos que $f^n(d)\sqsubseteq \dots \sqsubseteq f(d) \sqsubseteq d$ y que $f^{n}(\bot) \sqsubseteq f^n(d)$, para cada $n \in N$. Pero, entonces, 
$$\n{fix}(f) := \bigsqcup\{f^n(\bot) \, | \, n \geq 0\} \sqsubseteq \bigsqcup\{f^n(d) \, | \, n \geq 0\} \sqsubseteq d.$$
\end{proof}


\section{Estudio del punto fijo}


\subsection{Descripción}

Regresemos entonces a nuestro problema original. Para empezar, vamos a definir un orden parcial en $\mathbf{State}\hookrightarrow \mathbf{State}$, recordando los requisitos de punto fijo que ya expusimos anteriormente:
\begin{definition}
Dadas $g_1, g_2 \in \mathbf{State}\hookrightarrow \mathbf{State}$, decimos que $g_1$ \textit{está menos definida que} $g_2$ si, para arbitrarios $s, s' \in \mathbf{State}$, $g_1 s = s'$ implica que $g_2 s = s'$. Simbolizaremos esta relación como $\sqsubseteq$. 
\end{definition}
Evidentemente, la anterior definición se puede reformular mediante pares, de modo que dice que $\nn{G}(g_1)\subseteq G(g_2)$, donde $G(f)$ denota el \textit{grafo} de $f$, es decir, el conjunto de los pares de los que se compone $f$.

\begin{lema}
$(\mathbf{State}\hookrightarrow \mathbf{State}, \sqsubseteq)$ es conjunto parcialmente ordenado con elemento mínimo dado por $\bot \equiv \n{INDEFINIDO}$.
\end{lema}
\begin{proof}
Las propiedades reflexiva y transitiva son directas. Veamos la antisimétrica. Sean $g_1 \sqsubseteq g_2 \sqsubseteq g_1$. Si $g_1s = s'$, entonces evidentemente se sigue que $g_2s = s'$. Si $g_1s = \n{INDEFINIDO}$, entonces $g_2s = \n{INDEFINIDO}$ porque, en otro caso, digamos que $g_2s = s'$, tendríamos que $g_1s=s'$, lo que es imposible.

Veamos lo segundo. Evidentemente, $\bot \in \mathbf{State}\hookrightarrow \mathbf{State}$. Además, $\bot s = s'$ implica de forma vacía que $gs = s'$, para cada $g \in \mathbf{State}\hookrightarrow \mathbf{State}$.
\end{proof}

\begin{example}
Sean $g_1, g_2, g_3, g_4 \in \mathbf{State}\hookrightarrow \mathbf{State}$:
$$g_1 := id$$
$$g_2 s := \begin{cases}
    s \text{, si } sx \geq 0 \\
    \n{INDEFINIDO} \text{, en otro caso}
\end{cases}$$
$$g_3 s := \begin{cases}
    s \text{, si } sx = 0 \\
    \n{INDEFINIDO} \text{, en otro caso}
\end{cases}$$
$$g_4 s := \begin{cases}
    s \text{, si } sx \leq 0 \\
    \n{INDEFINIDO} \text{, en otro caso}
\end{cases}$$
Entonces es claro que tenemos el siguiente diagrama de Hasse: 
$$
\begin{tikzcd}
&g_1& \\
&& \\
g_2\arrow[uur]{} && g_4\arrow[uul]{} \\
&& \\
&g_3\arrow[uul]{} \arrow[uur]{}&
\end{tikzcd}
$$
Donde cada arista indica que se da la relación de orden entre los nodos (en el sentido especificado). No aparecen las aristas obtenidas mediante transitividad (dicho de otro modo, la transitividad viene dada por llegar de un nodo a otro mediante un camino válido).
\end{example}
Entonces, las propiedades que debía cumplir un punto fijo para un funcional $F$ quedan como: 
\begin{itemize}
    \item[(a)] $F(\n{fix}(F)) =  \n{fix}(F)$.
    \item[(b)] Si $g$ es punto fijo de $F$, $\n{fix}(F) \sqsubseteq g$.
\end{itemize}
Además, es claro que, si $F$ tiene al menos un $\sqsubseteq$-menor punto fijo, entonces es único, por la antisimetría del orden.
\\

El siguiente ejemplo muestra cómo podemos hablar de cadenas en $\mathbf{State}\hookrightarrow \mathbf{State}$:
\begin{example}
Consideremos la sucesión $(g_n)\subseteq \mathbf{State}\hookrightarrow \mathbf{State}$ dada por
$$g_n\ s := \begin{cases}
    \n{INDEFINIDO} \text{, si } s\ x > n \\
    s[\n{x} \mapsto -1] \text{, si } 0 \leq s\ x < n\\
    s \text{, si } s\ x < 0
\end{cases}$$
Notemos que $g_n \sqsubseteq g_m$ si $n \leq m$ porque $g_n$ está indefinida para más casos que $g_m$. Entonces, $C: \{g_n \, | \, n \in \N\}$ es cadena. $C$ está acotada superiormente por
$$g\ s := \begin{cases}
    s[\n{x} \mapsto -1] \text{, si } 0 \leq s\ x\\
    s \text{, si } s\ x < 0
\end{cases}$$
y, de hecho, $g$ es supremo de $C$.
\end{example}

Tenemos el siguiente resultado:

\begin{lema}
$(\mathbf{State}\hookrightarrow \mathbf{State}, \sqsubseteq)$ es conjunto parcialmente ordenado completo. Si $C$ es una cadena en $\mathbf{State}\hookrightarrow \mathbf{State}$, entonces\footnote{Nótese que esto quiere decir que $(\bigsqcup C)s = s'$ si y solo si $gs = s'$, para cierta $g \in C$.}
$$G\left(\bigsqcup C\right) = \bigcup\{G(g) \, | \, g \in C\}.$$
\end{lema}
\begin{proof}
Veamos que $\bigcup\{G(g) \, | \, g \in C\}$ es grafo de una función parcial en $\mathbf{State}\hookrightarrow \mathbf{State}$. Sean $(s, s'), (s, s'')$ elementos de este conjunto. Entonces existen $g, h \in C$ tales que $gs = s'$ y $hs = s''$. Como $C$ es cadena, $h$ y $g$ son comparables, luego obtenemos $s' = gs = hs = s''$, como queríamos. 
\\

Sea entonces $g_0$ tal función parcial con $G(g_0) = \bigcup\{G(g) \, | \, g \in C\}$. Veamos que $g_0$ es cota superior de $C$: como es obvio que $G(g) \subseteq G(g_0)$ para cada $g \in C$, tenemos que $g \sqsubseteq g_0$.
\\

Finalmente, veamos que $g_0$ es el supremo de $C$. En efecto, si $g_1$ es otra cota superior de $C$, tendremos que $G(g) \subseteq G(g_1)$, para cada $g \in C$. Tomando la unión en cada lado, tenemos que 
$$G(g_0) = \bigcup\{G(g) \, | \, g \in C\} \subseteq G(g_1)$$
es decir, $g_0 \sqsubseteq g_1$, como queríamos.
\end{proof}

Con el siguiente ejemplo vemos qué aspecto tiene una aplicación continua en el caso especial que estudiamos:
\begin{example}
Recordemos el siguiente funcional asociado a $\n{while} \neg(\n{x}=0)\n{do skip}$:
$$F(g)s := \begin{cases}
    gs \text{, si } s\n{x} \neq 0\\
    s \text{, en otro caso}
\end{cases}$$
Veamos que es continuo. Tenemos que ver que $F$ es monótona y que $F(\bigsqcup C) = \bigsqcup F(C)$ para una cadena arbitraria $C\neq \emptyset$. Sean $g \sqsubseteq h$ cualesquiera. Entonces, basta ver la definición para notar que la relación de $F(g)$ y $F(h)$ depende únicamente de la que ya había entre $g$ y $h$, porque, si $F(g)s = s'$, entonces tenemos que $F(g)s$ puede ser $gs = s' = hs$ o bien puede ser $s$, es decir, $F(g)\sqsubseteq F(h)$. 
\\

Por otro lado, sea $C \neq \emptyset$ cadena y sea $\bigsqcup C$. Consideremos $F(\bigsqcup C)$. Sean $a$
\end{example}

\subsection{Existencia}

El Teorema \ref{fixth} aplicado al orden $\sqsubseteq$ nos dice que, bajo determinadas condiciones, $\n{fix}(\cdot)$ permite definir una función. En esta sección queremos estudiar estas condiciones. Para empezar, recordemos que definíamos 
$$\mc{S}_\nn{ds} [[\nwhile b \ndo S]] :=  \n{fix}(F),$$
siendo $F(g) := \n{cond}(\fc{B}{b}, g\circ \mc{S}_\nn{ds}[[S]], id)$. Por tanto, debemos verificar que $F$ verifica las condiciones de las que hablamos. El siguiente Lema nos da un resultado en esta dirección:

\begin{lema}
Sea $p \in \mathbf{State}\to\mathbf{Bool}$. La aplicación $\Phi(g, h) := \n{cond}(p, g, h)$ es continua en las dos variables $g, h$, es decir, tanto
$$F(g) := \n{cond}(p, g, g_0) \quad \text{como} \quad H(g) := \n{cond}(p, g_0, g)$$
son continuas, siendo $g_0 \in \mathbf{State}\hookrightarrow\mathbf{State}$.
\end{lema}
\begin{proof}
Vamos a demostrar la continuidad de $F$, el otro caso es análogo. Primero veamos la monotonía de $F$. Sean $g_1 \sqsubseteq g_2$. Supongamos que $F(g_1)s = s'$. Si $ps = \mathbf{tt}$, $F(g_1)s = g_1s = s'$ implica $s' = g_2s = F(g_2)s$. Si $ps = \mathbf{ff}$, $F(g_1)s = g_0s = F(g_2)s$ y el resultado es trivial.
\\

Veamos ahora la continuidad. Basta ver que, si $C \subseteq \mathbf{State}\hookrightarrow\mathbf{State}$ es una cadena arbitraria no vacía, $F(\bigsqcup C) \sqsubseteq \bigsqcup F(C)$ (la otra dirección viene dada, en general, por monotonía). Por lo que ya vimos antes, podemos ver que $G(F(\bigsqcup C)) \subseteq \bigcup \{G(F(g)) \, | \, g \in C\}$ para tener el resultado. Sea $(s, s')\in G(F(\bigsqcup C))$. Si $ps = \mathbf{tt}$, $F(\bigsqcup C)s = g_0s= s'$. Pero entonces, para cada $g \in C$, $F(g)s = g_0s = s'$. Si $ps = \mathbf{tt}$, $F(\bigsqcup C)s = (\bigsqcup C)s = s'$, es decir, $(s, s')\in G(\bigsqcup C) = \bigcup \{G(g) \, | \, g \in C\}$, luego existe cierta $h \in C$ con $hs = s'$, es decir, $F(h)s = s'$, lo que nos da el resultado.
\end{proof}

\begin{lema}
La aplicación $\Psi(g, h) := g \circ h$ es continua en las dos variables, es decir, tanto
$$F(g) := g \circ g_0\quad \text{como}\quad H(g) := g_0 \circ g$$
son continuas, siendo $g_0 \in \mathbf{State}\hookrightarrow\mathbf{State}$.
\end{lema}
\begin{proof}
Veamos el caso $F$. $F$ es monótona: $g_1 \sqsubseteq g_2$ implica $G(g_1)\subseteq G(g_2)$, luego
$$G(g_0)\circ G(g_1) \subseteq G(g_0)\circ G(g_2)$$
que prueba que $F(g):=g_0\circ g_1\sqsubseteq g_0\circ g_2 =: F(g_2)$.
\\
$F$ es continua: sea $C$ cadena no vacía cualquiera. Entonces, 
$$G\left(F\left(\bigsqcup C\right)\right) = G\left(\bigsqcup C \circ g_0\right) = G(g_0)\circ G\left(\bigsqcup C\right)$$
donde hemos empleado $\circ$ en el segundo miembro como la composición de relaciones\footnote{Sean $R \subseteq A \times B, S \subseteq B \times C$. La \textit{composición} de $R$ y $S$ se define como
$$R \circ S := \{(x, z) \in A \times C \, | \, \exists y \in B.(xRy \land ySz)\}$$
Entonces, notemos que el orden de composición difiere del que tenemos para las funciones, es decir, $$G(f \circ g) = G(g) \circ G(f).$$}.
Ahora bien, lo anterior es igual a
$$G(g_0)\circ \bigcup \{G(g) \, | \, g \in C\} = \bigcup \{G(g_0)\circ G(g) \, | \, g \in C\} = G\left(\bigsqcup F(C)\right).$$
\end{proof}


\begin{theorem}
El sistema de semántica denotacional para $\n{While}$ define una función total $\mc{S}_\nn{ds}:\mathbf{Stm}\to (\mathbf{State}\hookrightarrow\mathbf{State})$.
\end{theorem}
\begin{proof}
Como tenemos definiciones composicionales, la demostración es por inducción estructural sobre la metavariable $S \in \mathbf{Stm}$. Los casos base son:
\begin{itemize}
    \item[(i)] Si $S$ es de la forma $x := a$, entonces basta comprobar que la aplicación $s \mapsto s[x \mapsto \fc{A}{a}s]$ está bien definida. Pero esto es directo.
    \item[(ii)] Si $S = \n{skip}$, la aplicación asociada es la identidad, que está bien definida. 
\end{itemize}
Los casos inductivos son:
\begin{itemize}
    \item[(i)] Si $S = S_1;S_2$, entonces $\mc{S}_\nn{ds}[[S]] = \mc{S}_\nn{ds}[[S_2]]\circ \mc{S}_\nn{ds}[[S_1]]$, que es composición, por hipótesis de inducción, de funciones bien definidas.
    \item[(ii)] Si $S = \n{if }b \n{ then }S_1 \n{ else }S_2$, de nuevo tenemos que $\mc{S}_\nn{ds}[[S_1]]$ y $\mc{S}_\nn{ds}[[S_2]]$ están bien definidas por hipótesis. Pero es directo comprobar que esto se mantiene por la aplicación $\n{cond}$.
    \item[(iii)] Si $S= \n{while }b \n{ do }S_1$, sabemos que $\mc{S}_\nn{ds}[[S_1]]$ está bien definida. Pero las aplicaciones
    $$F_1(g) := \n{cond}(\fc{B}{b}, g, id), \quad F_2(g) := g \circ \mc{S}_\nn{ds}[[S_1]]$$
    son continuas por los lemas previos. Como sabemos que composición de continuas es continua, tenemos que $F(g) := F_1(F_2(g))$ es continua y, por el Teorema \ref{fixth}, $\n{fix}(F)$ está bien definido, como queríamos ver.
\end{itemize}
\end{proof}

\begin{example} Veamos cómo aplica $\mc{S}_\nn{ds}$ a la sentencia
\[
    S \equiv \n{y}:=\n{1}; \nwhile \neg(\n{x}=1) \ndo (\n{y}:=\n{y}*\n{x}; \n{x}:=\n{x}-\n{1})
\]
y cómo obtener el punto fijo para su bucle. Aplicamos la reglas del sistema denotacional para While (\ref{whileds}) y nos sale
\[
    \mc{S}_\nn{ds}[[S]] s_0 = \n{fix}(F)\ s_0[\n{y} \mapsto \n{1}]
\]
donde
\[
    (F\ g)\ s = \left\{\begin{array}{ll}
         g\ (\mc{S}_\nn{ds}[[\n{y}*\n{x}; \n{x}:=\n{x}-\n{1}]]s) & \fc{B}{\neg(\n{x} = \n{1})}s = \textbf{tt} \\
         s & \fc{B}{\neg(\n{x} = \n{1})}s = \textbf{ff} 
    \end{array}\right.
\]
Aplicando nuevamente las reglas de $\mc{S}_\nn{ds}$ se puede reescribir como
\[
    (F\ g)\ s = \left\{\begin{array}{ll}
         g\ (s[\n{y}\mapsto (s\ \n{y})*(s\ \n{x})][\n{x}\mapsto(s\ \n{x})-\n{1}]) & s\ \n{x} \neq \n{1} \\
         s & s\ \n{x} = \n{1}
    \end{array}\right.
\]
Ahora computamos $F^n\ \bot$ para obtener el menor punto fijo
\begin{eqnarray*}
    (F^0 \bot)\ s &=& \n{INDEFINIDO} \\
    (F^1 \bot)\ s &=& \left\{\begin{array}{ll}
         \n{INDEFINIDO} & s\ \n{x} \neq 1 \\
         s & s\ \n{x} = 1
    \end{array}\right. \\
    (F^2 \bot)\ s &=& \left\{\begin{array}{ll}
         \n{INDEFINIDO} & s\ \n{x} \neq 1 \land s\ \n{x} \neq 2 \\
         s[\n{y}\mapsto (s\ \n{y})*2][x\mapsto 1] & s\ \n{x} = 2 \\
         s & s\ \n{x} = 1
    \end{array}\right. \\
\end{eqnarray*}
Cada evaluación de $F$ representa el cómputo del cuerpo del bucle tantas veces como la condición booleana se cumpla. La fórmula general es
\[
    (F^n \bot)\ s = \left\{\begin{array}{ll}
         \n{INDEFINIDO} & s\ \n{x} < 1 \lor s\ \n{x} > \n{n} \\
         s[\n{y}\mapsto (s\ \n{y})*j*...*2*1][x\mapsto 1] & s\ \n{x} = j \land 1\leq j \land j\leq n \\
    \end{array}\right.
\]
y se tiene que el punto fijo es
\[
    (\n{fix}\ F)\ s = \left\{\begin{array}{ll}
         \n{INDEFINIDO} & s\ \n{x} < 1 \\
         s[\n{y}\mapsto (s\ \n{y})*(s\ \n{x})!][x\mapsto 1] & s\ \n{x}\geq 1 \\
    \end{array}\right.
\]
Si se toma por ejemplo un estado $s_0\in\State$ tal que $s_0\ \n{x} = 3$ se tendría
\[
     (\n{fix}(F)\ s_0[\n{y} \mapsto \n{1}])\ \n{y} = 1*3*2*1 = 6
\]
\end{example}

\begin{example}
Consideremos $\mc{S}_\nn{ds}[[\n{while true do skip}]]$. Tenemos que, $F(g) := \n{cond}(\fc{B}{\n{true}}, g\circ\mc{S}_\nn{ds}[[\n{skip}]], id) =  \n{cond}(\fc{B}{\n{true}}, g\circ id, id) = g$, luego cada aplicación es punto fijo de $F$, y evidentemente, $\bot$ es el menor de todos ellos.

\end{example}
 
\subsection{Equivalencia semántica}

Ahora que ya disponemos de un sistema de semántica denotacional, parece lógico repetirse las mismas preguntas que ya nos hicimos en su momento con la semántica operacional.

\begin{definition}
Dos sentencias $S_1, S_2 \in \mathbf{Stm}$ se dicen \textit{semánticamente equivalentes} si $\mc{S}_\nn{ds}[[S_1]] = \mc{S}_\nn{ds}[[S_2]]$.
\end{definition}

\begin{example} 
EJERCICIO 5.53: Demostrar que los siguientes statements son enquivalentes:
\begin{itemize}
    \item $S; \n{skip  y S}$
    \item $S_1;(S_2;S_3) \n{ y } (S_1; S_2);S_3$ 
    \item $\n{while } b \n{ do } S $ y $\n{if } b \n{ then } (S; \n{while } b \n{ do } S ) \n{ else skip}  $
\end{itemize}
\end{example}

\begin{example}
EJERCICIO 5.54 Demostrar la equivalencia entre $\n{repeat } S \n{ until } b $ y $S; \n{while } \neg b \n{do} S $

$$ \mc{S}_\nn{S}[[ \n{repeat } S \n{ until } b ]] := \n{cond}(\fc{B}{b, \n{skip}}$$

\end{example}


\section{Teorema de equivalencia}

Como siempre, cuando introducimos un nuevo sistema de semántica, nos interesa estudiar si se encuentra relacionado con otros que ya hemos presentado antes. Más concretamente, nos interesa el siguiente enunciado:

\begin{theorem}
Para cada $S \in \mathbf{Stm}$, $\mc{S}_\nn{sos}[[S]] = \mc{S}_\nn{ds}[[S]]$.
\end{theorem}

Notemos que, tanto $\mc{S}_\nn{sos}[[S]]$ como $\mc{S}_\nn{ds}[[S]]$ son elementos de $\mathbf{State}\hookrightarrow\mathbf{State}$, que es un conjunto ordenado por $\sqsubseteq$, luego basta demostrar que $\mc{S}_\nn{sos}[[S]] \sqsubseteq \mc{S}_\nn{ds}[[S]]$ y $\mc{S}_\nn{ds}[[S]] \sqsubseteq \mc{S}_\nn{sos}[[S]]$, para $S \in \mathbf{Stm}$ arbitrario.

\appendix

%    Include appendix "chapters" here.


%    Bibliography styles amsplain or harvard are also acceptable.
\bibliographystyle{amsalpha}
%    See note above about multiple indexes.
\printindex

\end{document}